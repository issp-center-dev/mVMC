% !TEX root = userguide_jp.tex
%----------------------------------------------------------
\chapter{ファイル仕様}

%----------------------------------------------------------
\section{Input files for {\it Standard} mode}
\label{Ch:HowToStandard}

An example of input file for the standard mode is shown below:

\begin{minipage}{10cm}
\begin{screen}
\begin{verbatim}
W = 2
 L = 4
 model = "spin"

 lattice = "triangular lattice"
//mu = 1.0
// t = -1.0
// t' = -0.5
// U = 8.0
//V = 4.0
//V'=2.0
J = -1.0
J'=-0.5
// nelec = 8
2Sz = 0
\end{verbatim}
\end{screen}
\end{minipage}

{\bf Basic rules for input files}
\begin{itemize}
\item In each line, there is a set of a keyword (before an ``\verb|=|") and a parameter(after an ``\verb|=|"); 
  they are separated by ``\verb|=|".
\item You can describe keywords in a random order.
\item Empty lines and lines beginning in a ``\verb|//|''(comment outs) are skipped.
\item Upper- and lowercase are not distinguished.
  Double quotes and blanks are ignored.
\item There are three kinds of parameters.\\ 
  1.~Parameters that must be specified~(if not, \verb|vmcdry.out| will stop with error messages),\\ 
  2.~Parameters that is not necessary be specified~(if not, default values are used),\\
  3.~Parameters that must not be specified~(if specified, \verb|vmcdry.out| will stop with error messages).\\
  An example of 3 is transfer $t$ for the Heisenberg spin system. 
  If you choose ``model=spin", you should not specify ``$t$".
\end{itemize}

We explain each keywords as follows:

\subsection{Parameters about the kind of a calculation}

\begin{itemize}

\item \verb|model|

{\bf Type :} String (Choose from \verb|"Fermion Hubbard"|, \verb|"Spin"|, \verb|"Kondo Lattice"|, 
\verb|"Fermion HubbardGC"|, \verb|"SpinGC"|, \verb|"Kondo LatticeGC"|)
\footnote{GC=Grand Canonical}

{\bf Description :} The target model is specified with this parameter;
above words denote the canonical ensemble of the Fermion in the Hubbard model
\begin{align}
H = -\mu \sum_{i \sigma} c^\dagger_{i \sigma} c_{i \sigma} 
- \sum_{i \neq j \sigma} t_{i j} c^\dagger_{i \sigma} c_{j \sigma} 
+ \sum_{i} U n_{i \uparrow} n_{i \downarrow}
+ \sum_{i \neq j} V_{i j} n_{i} n_{j},
\label{fml4_1_hubbard}
\end{align}
canonical ensemble in the Spin model($\{\sigma_1, \sigma_2\}={x, y, z}$)
\begin{align}
H &= -h \sum_{i} S_{i z} - \Gamma \sum_{i} S_{i x} + D \sum_{i} S_{i z} S_{i z}
\nonumber \\
&+ \sum_{i j, \sigma_1}J_{i j \sigma_1} S_{i \sigma_1} S_{j \sigma_1}+ \sum_{i j, \sigma_1 \neq \sigma_2} J_{i j \sigma_1 \sigma_2} S_{i \sigma_1} S_{j \sigma_2} ,
\label{fml4_1_spin}
\end{align}
canonical ensemble in the Kondo lattice model
\begin{align}
H = - \mu \sum_{i \sigma} c^\dagger_{i \sigma} c_{i \sigma} 
- t \sum_{\langle i j \rangle \sigma} c^\dagger_{i \sigma} c_{j \sigma} 
+ \frac{J}{2} \sum_{i} \left\{
S_{i}^{+} c_{i \downarrow}^\dagger c_{i \uparrow}
+ S_{i}^{-} c_{i \uparrow}^\dagger c_{i \downarrow}
+ S_{i z} (n_{i \uparrow} - n_{i \downarrow})
\right\},
\label{fml4_1_kondo}
\end{align}
grand canonical ensemble of the Fermion in the Hubbard model [Eqn. (\ref{fml4_1_hubbard})],
grand canonical ensemble in the Spin model [Eqn. (\ref{fml4_1_spin})],
and
grand canonical ensemble in Kondo lattice model [Eqn. (\ref{fml4_1_kondo})]
respectively.

\item \verb|lattice|

{\bf Type :} String (Choose from \verb|"Chain Lattice"|, \verb|"Square Lattice"|, 
\verb|"Triangular Lattice"|, \verb|"Honeycomb Lattice"|, \verb|"Ladder"|, \verb|"Kagome"|)

{\bf Description :} The lattice shape is specified with this parameter;
above words denote
the one dimensional chain lattice (Fig. \ref{fig_chap04_1_lattice}(a)), 
the two dimensional square lattice (Fig. \ref{fig_chap04_1_lattice}(b)),
the two dimensional triangular lattice (Fig. \ref{fig_chap04_1_lattice}(c)),
the two dimensional anisotropic honeycomb lattice (Fig. \ref{fig_chap04_1_honeycomb}),
the ladder lattice (Fig. \ref{fig_ladder}),
and
the Kagome Lattice(Fig. \ref{fig_kagome})
respectively.

In \verb|method="SpinGCBoost"|,
only \verb|"Chain Lattice"|, \verb|"Honeycomb Lattice"|, 
\verb|"Ladder"|, and \verb|"Kagome"| are supported.
Limits of $L$, $W$, and the number of MPI processes ($N_{\rm proc}$) are as follows:

\begin{itemize}

  \item \verb|"Chain Lattice"|

    $L = 8n$ (where $n$ is an integer number under the condition  $n\geq1$),
    $N_{\rm proc} \leq 2(L=8)$, $N_{\rm proc} \leq 2^{L/2-2}(L>8)$.
    
  \item \verb|"Honeycomb Lattice"|

    $W=3, L \geq 2$, $N_{\rm proc} \leq 2(L=2)$, $N_{\rm proc} \leq 64(L>2)$.

  \item \verb|"Ladder"|

    $W=2, L = 2n$ (where $n$ is an integer number under the condition  $n\geq4$),
    $N_{\rm proc} \leq 2^{L-4}$.

  \item \verb|"Kagome"|

    $W=3, L \geq 2$, $N_{\rm proc} \leq 1(L=2)$, $N_{\rm proc} \leq 512(L>2)$.

\end{itemize}

\begin{figure}[!tbhp]
  \begin{center}
    \includegraphics[width=8cm]{../figs/chap04_1_lattice.pdf}
    \caption{Schematic illustration of
      (a) one dimensional chain lattice, 
      (b) two dimensional square lattice, and 
      (c) two dimensional triangular lattice.
      They have $t$, $V$, and $J$ as a nearest neighbor hopping, an offsite Coulomb integral, 
      and a spin-coupling constant, respectively (magenta solid lines);
      They also have $t'$, $V'$, and $J'$ as a next nearest neighbor hopping, offsite Coulomb integral, 
      and spin-coupling constant, respectively (green dashed line).
    }
    \label{fig_chap04_1_lattice}
  \end{center}
\end{figure}

\begin{figure}[!tbhp]
  \begin{center}
    \includegraphics[width=15cm]{../figs/chap04_1_honeycomb.pdf}
    \caption{Schematic illustration of the anisotropic honeycomb lattice.
      The nearest neighbor 
      hopping integral, spin coupling, offsite Coulomb integral
      depend on the bond direction.
      Those between second nearest neighbor sites are not supported.
    }
    \label{fig_chap04_1_honeycomb}
  \end{center}
\end{figure}

\begin{figure}[!tbhp]
  \begin{center}
    \includegraphics[width=10cm]{../figs/kagome.pdf}
    \caption{Schematic illustration of the Kagome lattice.
    }
    \label{fig_kagome}
  \end{center}
\end{figure}

\begin{figure}[!tbhp]
  \begin{center}
    \includegraphics[width=10cm]{../figs/ladder.pdf}
    \caption{Schematic illustration of the ladder lattice.
    }
    \label{fig_ladder}
  \end{center}
\end{figure}

\end{itemize}

\subsection{Parameters for the lattice}

\subsubsection{Chain [Fig. \ref{fig_chap04_1_lattice}(a)]}

\begin{itemize}

\item \verb|L|

{\bf Type :} Integer

{\bf Description :} The length of the chain is specified 
with this parameter.

\end{itemize}

\subsubsection{Ladder (Fig. \ref{fig_ladder})}

\begin{itemize}

\item \verb|L|

{\bf Type :} Integer

{\bf Description :} The length of the ladder is specified 
with this parameter.

\item \verb|W|

{\bf Type :} Integer

{\bf Description :} The number of the ladder is specified 
with this parameter.

\end{itemize}

\begin{figure}[!tbhp]
  \begin{center}
    \includegraphics[width=15cm]{../figs/chap04_1_unitlattice.pdf}
    \caption{The shape of the numerical cell 
      when ${\vec a}_0 = (6, 2), {\vec a}_1 = (2, 4)$
      in the triangular lattice.
      The region surrounded by 
      ${\vec a}_0$(Magenta dashed arrow) and ${\vec a}_1$(Green dashed arrow)
      becomes the cell to be calculated (20 sites).
    }
    \label{fig_chap04_1_unitlattice}
  \end{center}
\end{figure}

\subsubsection{Tetragonal lattice [Fig. \ref{fig_chap04_1_lattice}(b)], 
Triangular lattice[Fig. \ref{fig_chap04_1_lattice}(c)],
Honeycomb lattice(Fig. \ref{fig_chap04_1_honeycomb}),
Kagome lattice(Fig. \ref{fig_kagome})}

In these lattices,
we can specify the shape of the numerical cell by using the following two methods.

\begin{itemize}

\item \verb|W|, \verb|L|

{\bf Type :} Integer

{\bf Description :} The alignment of original unit cells 
(dashed black lines in Figs. \ref{fig_chap04_1_lattice} - \ref{fig_kagome})
is specified with these parameter.

\item \verb|a0W|, \verb|a0L|, \verb|a1W|, \verb|a1L|

{\bf Type :} Integer

{\bf Description :} 
We can specify two vectors (${\vec a}_0, {\vec a}_1$)
that surrounds the numerical cell (Fig. \ref{fig_chap04_1_unitlattice}).
These vectors should be specified in the Fractional coordinate.

\end{itemize}

If we use both of these method, \verb|vmcdry.out| stops.

We can check the shape of the numerical cell
by using a file \verb|lattice.gp|
which is written in the Standard mode.
This file can be read by \verb|gnuplot| as follows:
\begin{verbatim}
$ gnuplot lattice.gp
\end{verbatim}

\subsection{副格子}

以下パラメータを用いると変分波動関数のペア軌道部分に副格子の周期性を持たせることが出来ます。

\begin{itemize}

\item \verb|a0Wsub|, \verb|a0Lsub|, \verb|a1Wsub|, \verb|a1Lsub|, \verb|Wsub|, \verb|Lsub|

{\bf 形式 :} 自然数。デフォルトでは
\verb|a0Wsub=a0W|, \verb|a0Lsub=a0L|, \verb|a1Wsub=a1W|, \verb|a1Lsub=a1L|, 
\verb|Wsub=W|, \verb|Lsub=L|となる。

{\bf 説明 :} これらのパラメーターの指定の仕方は
\verb|a0W|, \verb|a0L|, \verb|a1W|, \verb|a1L|, \verb|W|, \verb|L|
と同様です。
ただし、元の計算セルが副格子に整合しない場合にはプログラムを終了します。

\end{itemize}


\subsection{Parameters for the Hamiltonian}
A default value is set as $0$ unless a specific value is not defined in a description. 
Table~\ref{table_interactions} shows the parameters for each models. 
In the case of a complex type, a file format is ``{\it a real part, an imaginary part} "
 while in the case of a real type, only ``{\it a real part} ".


\subsubsection{Local terms}

\begin{itemize}

\item \verb|mu|

{\bf Type :} Real

{\bf Description :} (Hubbard and Kondo lattice model) 
The chemical potential $\mu$ (including the site potential)
is specified with this parameter.

\item \verb|U|

{\bf Type :} Real

{\bf Description :} (Hubbard and Kondo lattice model) 
The onsite Coulomb integral $U$ is specified with this parameter.

\item \verb|Jx|, \verb|Jy|, \verb|Jz|, \verb|Jxy|, 
  \verb|Jyx|, \verb|Jxz|, \verb|Jzx|, \verb|Jyz|, \verb|Jzy|

{\bf Type :} Real

{\bf Description :} (Kondo lattice model) 
The spin-coupling constant between the valence and the local electrons
is specified with this parameter.
If the exchange coupling \verb|J| is specified in the input file,
instead of \verb|Jx, Jy, Jz|,
the diagonal exchange couplings, \verb|Jx, Jy, Jz|, are set as \verb|Jx = Jy = Jz = J|.
When both
the set of exchange couplings (\verb|Jx|, \verb|Jy|, \verb|Jz|)
and the exchange coupling \verb|J| are specified in the input file,
\verb|vmcdry.out| will stop.

\item \verb|h|, \verb|Gamma|, \verb|D|

{\bf Type :} Real

{\bf Description :} (Spin model)
The longitudinal magnetic field, transverse magnetic field, 
and the single-site anisotropy parameter are specified with these parameters.
The single-site anisotropy parameter is not available for \verb|model=SpinGCBoost|.

\end{itemize}

The non-local terms described below should be specified
in different ways depending on the lattice structure:
For \verb|lattice=Ladder|, the non-local terms are specified in the different way
from \verb|lattice=Chain Lattice|, \verb|Square Lattice|, \verb|Triangular Lattice|, \verb|Honeycomb Lattice|, \verb|Kagome|. 
Below, the available parameters for each lattice are shown in
Table \ref{table_interactions}.

\begin{table}[tbhp]
  \begin{tabular}{|l||c|c|c|c|c|c|c|c|} \hline
    Interactions & 1D chain & 2D square & 2D triangular & Honeycomb & Kagome & Ladder\\ 
    \hline 
    \hline
     \verb|J|, \verb|t|, \verb|V| (simplified) & $\circ$	 & $\circ$ & $\circ$ & $\circ$ & $\circ$ & -\\ 
     \hline
    \verb|J'|, \verb|t'|, \verb|V'| & $\circ$	 & $\circ$	& $\circ$ 	& $\circ$ 	& $\circ$ & - \\ 
    \hline
    \verb|J0|, \verb|t0|, \verb|V0| & $\circ$  & $\circ$ 	& $\circ$ 	& $\circ$ 	& $\circ$ & $\circ$\\ 
    \hline
    \verb|J1|, \verb|t1|, \verb|V1| & -         	 & $\circ$ 	& $\circ$ 	& $\circ$ 	& $\circ$ & $\circ$\\ 
    \hline
    \verb|J2|, \verb|t2|, \verb|V2|  & -         	 & -    	& $\circ$ 	& $\circ$ 	& $\circ$ & $\circ$\\
    \hline
    \verb|J1'|, \verb|t1'|, \verb|V1'| & -		 &-	 	& -		& -		& -		& $\circ$\\
    \hline
    \verb|J2'| ,\verb|t2'|, \verb|V2'|  & -		 &-	 	& -		& -		& -		& $\circ$\\ 
    \hline
\end{tabular}
   \caption{Interactions for each models defined in an input file. We can define spin couplings as matrix format.}
    \label{table_interactions}
\end{table}

\subsubsection{Non-local terms[ for Ladder (Fig. \ref{fig_ladder})]}

\begin{itemize}
\item \verb|t0|,  \verb|t1|,  \verb|t1'|,  \verb|t2|,  \verb|t2'|

{\bf Type :} Complex

{\bf Description :} (Hubbard and Kondo lattice model)
Hopping integrals in the ladder lattice 
(See Fig. \ref{fig_ladder}) is specified with this parameter.

\item \verb|V0|,  \verb|V1|,  \verb|V1'|,  \verb|V2|,  \verb|V2'|

{\bf Type :} Real

{\bf Description :} (Hubbard and Kondo lattice model)
Offsite Coulomb integrals on the ladder lattice
(Fig. \ref{fig_chap04_1_honeycomb} are specified with these parameters.

\item \verb|J0x|, \verb|J0y|, \verb|J0z|, \verb|J0xy|, 
  \verb|J0yx|, \verb|J0xz|, \verb|J0zx|, \verb|J0yz|, \verb|J0zy|
\item \verb|J1x|, \verb|J1y|, \verb|J1z|, \verb|J1xy|, 
  \verb|J1yx|, \verb|J1xz|, \verb|J1zx|, \verb|J1yz|, \verb|J1zy|
\item \verb|J1'x|, \verb|J1'y|, \verb|J1'z|, \verb|J1'xy|, 
  \verb|J1'yx|, \verb|J1'xz|, \verb|J1'zx|, \verb|J1'yz|, \verb|J1'zy|
\item \verb|J2x|, \verb|J2y|, \verb|J2z|, \verb|J2xy|, 
  \verb|J2yx|, \verb|J2xz|, \verb|J2zx|, \verb|J2yz|, \verb|J2zy|
\item \verb|J2'x|, \verb|J2'y|, \verb|J2'z|, \verb|J2'xy|, 
  \verb|J2'yx|, \verb|J2'xz|, \verb|J2'zx|, \verb|J2'yz|, \verb|J2'zy|

{\bf Type :} Real

{\bf Description :} (Spin model)
Spin-coupling constants in the ladder lattice
(See Fig. \ref{fig_ladder}) are specified with these parameter.
If the simplified parameter \verb|J0| is specified in the input file instead of
the diagonal couplings, \verb|J0x, J0y, J0z|,
these diagonal couplings are set as \verb|J0x = J0y = J0z = J0|.
If both \verb|J0| and the set of the couplings (\verb|J0x, J0y, J0z|)
are specified, \verb|vmcdry.out| will stop.
The above rules are also valid for the simplified parameters, \verb|J1|, \verb|J1'|, \verb|J2|, and \verb|J2'|.

\end{itemize}

\subsubsection{Non-local terms [other than Ladder (Figs. \ref{fig_chap04_1_lattice}, \ref{fig_chap04_1_honeycomb},
\ref{fig_kagome})]}

\begin{itemize}
\item \verb|t0|,  \verb|t1|, \verb|t2|

{\bf Type :} Complex

{\bf Description :} (Hubbard and Kondo lattice model)
Nearest neighbor hoppings for each direction
(See Figs. \ref{fig_chap04_1_lattice}-\ref{fig_kagome})
are specified with these parameter.
If there is no bond dependence of the hoppings,
the simplified parameter \verb|t| is available to specify \verb|t0|,  \verb|t1|, and \verb|t2| as
\verb|t0 = t1 = t2 = t|.
If both \verb|t| and the set of the hoppings (\verb|t0|,  \verb|t1|, \verb|t2|) are specified,
\verb|vmcdry.out| will stop.

\item \verb|V0|,  \verb|V1|, \verb|V2|

{\bf Type :} Real

{\bf Description :} (Hubbard and Kondo lattice model)
Nearest-neighbor offsite Coulomb integrals $V$
 for each direction
(See Figs. \ref{fig_chap04_1_lattice}-\ref{fig_kagome})
are specified with these parameters.
If there is no bond dependence of the offsite Coulomb integrals,
the simplified parameter \verb|V| is available to specify \verb|V0|,  \verb|V1|, and \verb|V2| as
\verb|V0 = V1 = V2 = V|.
If both \verb|V| and the set of the Coulomb integrals (\verb|V0|,  \verb|V1|, \verb|V2|) are specified,
\verb|vmcdry.out| will stop.

\item \verb|J0x|, \verb|J0y|, \verb|J0z|, \verb|J0xy|, 
  \verb|J0yx|, \verb|J0xz|, \verb|J0zx|, \verb|J0yz|, \verb|J0zy|
\item \verb|J1x|, \verb|J1y|, \verb|J1z|, \verb|J1xy|, 
  \verb|J1yx|, \verb|J1xz|, \verb|J1zx|, \verb|J1yz|, \verb|J1zy|
\item \verb|J2x|, \verb|J2y|, \verb|J2z|, \verb|J2xy|, 
  \verb|J2yx|, \verb|J2xz|, \verb|J2zx|, \verb|J2yz|, \verb|J2zy|

{\bf Type :} Real

{\bf Description :} (Spin model)
Nearest-neighbor exchange couplings for each direction
are specified with thees parameters.
If the simplified parameter \verb|J0| is specified, instead of \verb|J0x, J0y, J0z|,
the exchange couplings, \verb|J0x, J0y, J0z|, are set as \verb|J0x = J0y = J0z = J0|.
If both \verb|J0| and the set of the exchange couplings (\verb|J0x, J0y, J0z|)
are specified, \verb|vmcdry.out| will stop.
The above rules are valid for \verb|J1| and \verb|J2|.

If there is no bond dependence of the exchange couplings,
the simplified parameters,
\verb|Jx|, \verb|Jy|, \verb|Jz|, \verb|Jxy|, 
\verb|Jyx|, \verb|Jxz|, \verb|Jzx|, \verb|Jyz|, \verb|Jzy|,
are available to specify the exchange couplings for every bond as
\verb|J0x = J1x = J2x = Jx|.
If any simplified parameter (\verb|Jx|$\sim$\verb|Jzy|)
is specified in addition to its counter parts (\verb|J0x|$\sim$\verb|J2zy|),
\verb|vmcdry.out| will stop.
Below, examples of parameter sets for nearest-neighbor exchange couplings are shown.

\begin{itemize}

\item If there are no bond-dependent, no anisotropic and offdiagonal exchange couplings (such as $J_{x y}$),
please specify \verb|J| in the input file.

\item If there are no bond-dependent and offdiagonal exchange couplings
but are anisotropic couplings,
please specify the non-zero couplings in the diagonal parameters, \verb|Jx, Jy, Jz|.

\item If there are no bond-dependent exchange couplings
but are anisotropic and offdiagonal exchange couplings,
please specify the non-zero couplings in the nine parameters,
\verb|Jx, Jy, Jz, Jxy, Jyz, Jxz, Jyx, Jzy, Jzx|.

\item If there are no anisotropic and offdiagonal exchange couplings,
but are bond-dependent couplings,
please specify the non-zero couplings in the three parameters,
\verb|J0, J1, J2|.

\item If there are no anisotropic exchange couplings, but are bond-dependent and offdiagonal couplings,
please specify the non-zero couplings in the nine parameters,
\verb|J0x, J0y, J0z, J1x, J1y, J1z, J2x, J2y, J2z|.

\item If there are bond-dependent, anisotropic and offdiagonal exchange couplings,
please specify the non-zero couplings in the twenty-seven parameters from
\verb|J0x| to \verb|J2zy|.

\end{itemize}
\item \verb|t'|

{\bf Type :} Complex

{\bf Description :} (Hubbard and Kondo lattice model)
Nearest neighbor hoppings for each direction
(See Figs. \ref{fig_chap04_1_lattice}-\ref{fig_kagome})
are specified with these parameter.

\item \verb|V'|

{\bf Type :} Real

{\bf Description :} (Hubbard and Kondo lattice model)
Nearest neighbor-offsite Coulomb integrals $V$
 for each direction
(See Figs. \ref{fig_chap04_1_lattice}-\ref{fig_kagome})
are specified with these parameters.

\item \verb|J'x|, \verb|J'y|, \verb|J'z|, \verb|J'xy|, 
  \verb|J'yx|, \verb|J'xz|, \verb|J'zx|, \verb|J'yz|, \verb|J'zy|

{\bf Type :} Real

{\bf Description :} (Spin model)
Second nearest-neighbor exchange couplings are specified.
However, for \verb|lattice = Honeycomb Lattice| and  \verb|lattice = Kagome|
with \verb|model=SpinGCBoost|,
the second nearest-neighbor exchange couplings are not available in the $Standard$ mode.
If the simplified parameter \verb|J'| is specified, instead of
\verb|J'x, J'y, J'z|,
the exchange couplings are set as
\verb|J'x = J'y = J'z = J'|.
If both \verb|J'| and the set of the couplings (\verb|J'x, J'y, J'z|),
\verb|vmcdry.out| will stop.

\item \verb|phase0|, \verb|phase1|

  {\bf 形式 :} 複素数(デフォルトでは\verb|1.0|)
  
  {\bf 説明 :} 計算するセルの境界をまたいだホッピング項に付く位相因子を指定することが出来ます。
  $\vec{a}_0$方向、$\vec{a}_1$方向それぞれ別の位相因子を用いることが出来ます。
  1次元系では\verb|phase0|のみ使用できます。
  例えば、$i$サイトから$i+1$サイトへのホッピングで、
  正の方向に境界をまたいだ場合には次のようになります。
  \begin{align}
    ({\rm phase0}) \times t {\hat c}_{i+1 \sigma}^\dagger {\hat c}_{i \sigma}
    + ({\rm phase0})^* \times t^* {\hat c}_{i \sigma}^\dagger {\hat c}_{i+1 \sigma}
  \end{align}

\end{itemize}

\subsection{計算条件のパラメーター}

\begin{itemize}

\item \verb|OutputMode|

  {\bf 形式 :} \verb|"none"|, \verb|"correlation"|, \verb|"full"|のいずれか(デフォルトは\verb|correlation|)

  {\bf 説明 :} 計算を行う相関関数を指定します。
\verb|"none"|の場合は相関関数を計算しません。
\verb|"correlation"|を指定した場合には、1体部分はすべての$i, \sigma$について
$\langle c_{i \sigma}^{\dagger}c_{i \sigma} \rangle$を、
2体部分はすべての$i, j, \sigma, \sigma'$について
$\langle c_{i \sigma}^{\dagger}c_{i \sigma} c_{j \sigma'}^{\dagger}c_{j \sigma'} \rangle$
を計算します。
\verb|"full"|を指定した場合には、
1体部分はすべての$i, j, \sigma, \sigma'$について
$\langle c_{i \sigma}^{\dagger}c_{j \sigma'} \rangle$を、
2体部分はすべての$i_1, i_2, i_3, i_4, \sigma_1, \sigma_2, \sigma_3, \sigma_4$について
$\langle c_{i_1 \sigma_1}^{\dagger}c_{i_2 \sigma_2} c_{i_3 \sigma_3}^{\dagger}c_{i_4 \sigma_4} \rangle$
を計算します。
スピン系の演算子はBogoliubov表現により生成消滅演算子で表されています。
詳しくは\ref{sec_bogoliubov_rep}をご覧ください。

  \item  \verb|CDataFileHead|

 {\bf 形式 :} string型 (空白不可、必須)

{\bf 説明 :} アウトプットファイルのヘッダ。例えば、一体のGreen関数の出力ファイル名が{\bf xxx\_cisajs.dat}として出力されます(xxxに\verb|CDataFileHead|で指定した文字が記載)。

 \item  \verb|CParaFileHead|

 {\bf 形式 :} string型 (空白不可、必須)

{\bf 説明 :} 最適化された変分パラメータの出力ファイル名のヘッダ。最適化された変分パラメータが{\bf xxx\_opt.dat}ファイルとして出力されます(xxxに\verb|CParaFileHead|で指定した文字が記載)。
 
 
 \item  \verb|NVMCCalMode|

 {\bf 形式 :} int型 (\tr{デフォルト値 = 0})

{\bf 説明 :} [0] 変分パラメータの最適化、[1] 1 体・2 体のグリーン関数の計算。
 
 \item  \verb|NLanczosMode|

 {\bf 形式 :} int型 (\tr{デフォルト値 = 0})

{\bf 説明 :} [0] 何もしない、[1] Single Lanczos Step でエネルギーまで計算、
[2] Single Lanczos Step で1 体・2 体のグリーン関数まで計算
(条件: 1, 2 は\verb|NVMCCalMode| = 1のみ使用可能. 
また, pair hopping項, exchange 項がハミルトニアンに含まれる場合は使用できません)。
 
 \item  \verb|NDataIdxStart|

 {\bf 形式 :} int型 (\tr{デフォルト値 = 0})

{\bf 説明 :} 出力ファイルの付加番号。
\verb|NVMCCalMode|= 0 の場合は\verb|NDataIdxStart|が出力され、 
\verb|NVMCCalMode| = 1 の場合は、
\verb|NDataIdxStart|から連番で\verb|NDataQtySmp|個のファイルを出力します。
   
 \item  \verb|NDataQtySmp|

 {\bf 形式 :} int型 (\tr{デフォルト値 = 1})

{\bf 説明 :} 出力ファイルのセット数。 \verb|NVMCCalMode| = 1 の場合に使用します。

\item  \verb|nelec|

  {\bf 形式 :} {int型 (1以上、必須)}

  {\bf 説明 :} \tr{伝導電子の数。
    $\uparrow$電子と$\downarrow$電子の個数を足したものを入力してください。}

\item  \verb|NSPGaussLeg|

  {\bf 形式 :} {int型 (1以上、\tr{デフォルト値 = 8})}

  {\bf 説明 :} スピン量子数射影の$\beta$積分($S^y$回転)のGauss-Legendre求積法の分点数。


\item  \verb|NSPStot|

  {\bf 形式 :} int型 (0以上、\tr{デフォルト値 = 0})

  {\bf 説明 :}  スピン量子数。

\item  \verb|NMPTrans|

  {\bf 形式 :} int型 (1以上、\tr{デフォルト値は副格子内部の並進ベクトルの数})

  {\bf 説明 :} 
  運動量・格子対称性の量子数射影の個数。
  TransSymファイルで指定した重みで上から\verb|NMPTrans|個まで使用する。\tr{射影を行わない場合は1に設定する。}

 \item  \verb|NSROptItrStep|

{\bf 形式 :} int型 (1以上、\tr{デフォルト値 = 1000})

{\bf 説明 :} 
SR 法で最適化する場合の全ステップ数。\verb|NVMCCalMode|=0の場合のみ使用されます。
 
 \item  \verb|NSROptItrSmp|

{\bf 形式 :} int型 (1以上数、\tr{デフォルト値 = }\verb|NSROptItrStep|/10)

{\bf 説明 :} \verb|NSROptItrStep|ステップ中、最後の\verb|NSROptItrSmp|ステップでの各変分パラメータの平均値を最適値とする。\verb|NVMCCalMode|=0の場合のみ使用されます。

\item   \verb|DSROptRedCut|
   
{\bf 形式 :} double型 (\tr{デフォルト値 = 0.001})

{\bf 説明 :} SR 法安定化因子。手法論文の$\varepsilon_{\rm wf}$に対応。

 \item  \verb|DSROptStaDel| 
   
 {\bf 形式 :} double型 (\tr{デフォルト値 = 0.02})

  {\bf 説明 :} SR 法安定化因子。手法論文の$\varepsilon$に対応。
     
\item \verb|DSROptStepDt|

{\bf 形式 :} double型 (\tr{デフォルト値 = 0.02})

{\bf 説明 :} \tr{要確認(マニュアルに説明なし)}
 
\item \verb|NVMCWarmUp|

{\bf 形式 :} int型 (1以上、\tr{デフォルト値=10})

{\bf 説明 :}マルコフ連鎖の空回し回数。

\item \verb|NVMCIniterval|

{\bf 形式 :} int型 (1以上、\tr{デフォルト値=1})

{\bf 説明 :} サンプル間のステップ間隔。ローカル更新が\verb|Nsite|× \verb|NVMCIniterval| 回行わます。

\item \verb|NVMCSample|

{\bf 形式 :} int型 (1以上、\tr{デフォルト値=1000})

{\bf 説明 :} 期待値計算に使用するサンプル数。

\item \verb|RndSeed|

{\bf 形式 :} int型 

{\bf 説明 :} 乱数の初期seed。\tr{MPI 並列では各計算機に}\verb|RndSeed|\tr{+my rank+1 で初期seed が与えられます。}

 \item \verb|NSplitSize|

{\bf 形式 :} int型 (1以上、\tr{デフォルト値=1})

{\bf 説明 :} mpi内部並列を行う場合の並列数。

\item \verb|NStoreOO|

{\bf 形式 :} int型 (1以上、\tr{デフォルト値=1})

{\bf 説明 :} 期待値$\langle O_k O_l \rangle$を計算するとき行列-行列積にするオプション(1で機能On)。
 
\item  \verb|ComplexType|
  
  {\bf 形式 :} int型 (\verb|0|もしくは\verb|1|、デフォルト値\verb|0|)

  {\bf 説明 :} \verb|0|のとき変分パラメータの実部のみを、\verb|1|のとき実部/虚部両方を最適化します。

\end{itemize}


\newpage
\section{エキスパートモード用入力ファイル}
\label{Ch:HowToExpert}

mVMCのエキスパートモードで使用する入力ファイル(*def)に関して説明します。
入力ファイルの種別は以下の4つで分類されます。
\begin{description}
\item[(1)~List:]
~\\{キーワード指定なし}:
使用するinput fileの名前のリストを書きます。なお、ファイル名は任意に指定することができます。
\item[(2)~Basic parameters:]
~\\{\bf ModPara}: 計算時に必要な基本的なパラメーター(サイトの数、電子数、Lanczosステップを何回やるかなど)を設定します。
~\\{\bf LocSpin}: 局在スピンの位置を設定します(近藤模型でのみ利用)。
\item[(3)~Set Hamiltonian:] 
~\\以下のファイルを用い、Hamiltonianを電子系の表式により指定します。
~\\{\bf Trans}: $c_{i\sigma_1}^{\dag}c_{j\sigma_2}$で表される一体項を指定します。
~\\{\bf InterAll}: $c_ {i \sigma_1}^{\dag}c_{j\sigma_2}c_{k \sigma_3}^{\dag}c_{l \sigma_4}$で表される一般二体相互作用を指定します。\\
~\\なお、使用頻度の高い相互作用に関しては下記のキーワードで指定することも可能です。
~\\{\bf CoulombIntra}: $n_ {i \uparrow}n_{i \downarrow}$で表される相互作用を指定します($n_{i \sigma}=c_{i\sigma}^{\dag}c_{i\sigma}$)。
~\\{\bf CoulombInter}: $n_ {i}n_{j}$で表される相互作用を指定します($n_i=n_{i\uparrow}+n_{i\downarrow}$)。
~\\{\bf Hund}: $n_{i\uparrow}n_{j\uparrow}+n_{i\downarrow}n_{j\downarrow}$で表される相互作用を指定します。
~\\{\bf PairHop}:  $c_ {i \uparrow}^{\dag}c_{j\uparrow}c_{i \downarrow}^{\dag}c_{j  \downarrow}$で表される相互作用を指定します。
~\\{\bf Exchange}: $S_i^+ S_j^-$で表される相互作用を指定します。
\item[(4)~Set condition of variational parameters :] 
~\\変分波動関数は
\begin{equation}
|\psi \rangle = {\cal P}_G{\cal P}_J{\cal P}_{d-h}^{(2)}{\cal P}_{d-h}^{(4)}{\cal L}^S{\cal L}^K{\cal L}^P |\phi_{\rm pair} \rangle,
\end{equation}
で与えられます。ここで、一体部分は実空間のペア関数
\begin{equation}
|\phi_{\rm pair} \rangle = \left[\sum_{i, j=1}^{N_s} f_{ij}c_{i\uparrow}^{\dag}c_{j\downarrow}^{\dag} \right]^{N/2}|0 \rangle,
\end{equation}
を用いた波動関数で表されます。ここで$N$は全電子数、$N_s$は全サイト数です。
変分パラメータの初期値は以下のファイルを用いて指定します。
~\\{\bf Gutzwiller}: ${\cal P}_G=\exp\left[ \sum_i g_i n_{i\uparrow} n_{i\downarrow} \right]$のうち、最適化の対象とする変分パラメータ$g_i$を指定します。
~\\{\bf Jastrow}: ${\cal P}_J=\exp\left[\frac{1}{2} \sum_{i\neq j} v_{ij} n_i n_j\right]$のうち、最適化の対象とする変分パラメータ$v_{ij}$を指定します。
~\\{\bf DH2}:  ${\cal P}_{d-h}^{(2)}= \exp \left[ \sum_t \sum_{n=0}^2 (\alpha_{2nt}^d \sum_{i}\xi_{i2nt}^d+\alpha_{2nt}^h \sum_{i}\xi_{i2nt}^h)\right]$で表される2サイトのdoublon-holon相関因子を指定します。詳細はDH2ファイルの説明を参照してください。
~\\{\bf DH4}:  ${\cal P}_{d-h}^{(4)}= \exp \left[ \sum_t \sum_{n=0}^4 (\alpha_{4nt}^d \sum_{i}\xi_{i4nt}^d+\alpha_{4nt}^h \sum_{i}\xi_{i4nt}^h)\right]$で表される4サイトのdoublon-holon相関因子を指定します。詳細はDH4ファイルの説明を参照してください。
~\\{\bf Orbital}: ペア軌道$|\phi_{\rm pair} \rangle = \left[\sum_{i, j=1}^{N_s} f_{ij}c_{i\uparrow}^{\dag}c_{j\downarrow}^{\dag} \right]^{N/2}|0 \rangle$を設定します。
~\\{\bf TransSym}: 運動量射影${\cal L}_K=\frac{1}{N_s}\sum_{{\bm R}}e^{i {\bm K} \cdot{\bm R} } \hat{T}_{\bm R}$と格子対称性射影${\cal L}_P=\sum_{\alpha}p_{\alpha} \hat{G}_{\alpha}$に関する指定を行います。ここで、${\bm K}$は全運動量、$\hat{T}_{\bm R}$は並進ベクトル${\bm R}$に対応する並進演算子、$\hat{G}_{\alpha}$は格子の点群演算子、$p_\alpha$はパリティをそれぞれ表します。

\item[(5)~Initial variational parameters:]
~\\ 変分パラメータに関する初期値を与えます。キーワード指定されない場合には$0$が初期値として設定されます。
~\\{\bf InGutzwiller}: ${\cal P}_G=\exp\left[ \sum_i g_i n_{i\uparrow} n_{i\downarrow} \right]$のうち、変分パラメータ$g_i$の初期値を指定します。
~\\{\bf InJastrow}: ${\cal P}_J=\exp\left[\frac{1}{2} \sum_{i\neq j} v_{ij} n_i n_j\right]$のうち、変分パラメータ$v_{ij}$の初期値を指定します。
~\\{\bf InDH2}:  ${\cal P}_{d-h}^{(2)}= \exp \left[ \sum_t \sum_{n=0}^2 (\alpha_{2nt}^d \sum_{i}\xi_{i2nt}^d+\alpha_{2nt}^h \sum_{i}\xi_{i2nt}^h)\right]$で表される2サイトのdoublon-holon相関因子$\alpha_{2nt}^{d(h)}$の初期値を指定します。
~\\{\bf InDH4}:  ${\cal P}_{d-h}^{(4)}= \exp \left[ \sum_t \sum_{n=0}^4 (\alpha_{4nt}^d \sum_{i}\xi_{i4nt}^d+\alpha_{4nt}^h \sum_{i}\xi_{i4nt}^h)\right]$で表される4サイトのdoublon-holon相関因子$\alpha_{4nt}^{d(h)}$の初期値を指定します。
~\\{\bf InOrbital}: ペア軌道$|\phi_{\rm pair} \rangle = \left[\sum_{i, j=1}^{N_s} f_{ij}c_{i\uparrow}^{\dag}c_{j\downarrow}^{\dag} \right]^{N/2}|0 \rangle$の$ f_{ij}$に関する初期値を設定します。


\item[(6)~Output:]
~\\{\bf OneBodyG }:出力する一体Green関数を指定します。
 $\langle c^{\dagger}_{i\sigma_1}c_{j\sigma_2}\rangle$が出力されます。

 {\bf TwoBodyG }:出力する二体Green関数を指定します。
 $\langle c^{\dagger}_{i\sigma_1}c_{j\sigma_2}c^{\dagger}_{k \sigma_3}c_{l\sigma_4}\rangle$
が出力されます。
\end{description}
%%%%%%%%%%%%%%%%%%%%%%
\newpage
~\subsection{入力ファイル指定用ファイル}
\label{Subsec:InputFileList}
計算で使用する入力ファイル一式を指定します。ファイル形式に関しては、以下のようなフォーマットをしています。\\
\begin{minipage}{10cm}
\begin{screen}
\begin{verbatim}
ModPara  modpara.def
LocSpin  zlocspn.def
Trans    ztransfer.def
InterAll zinterall.def
Orbital orbitalidx.def
OneBodyG zcisajs.def
TwoBodyG	zcisajscktaltdc.def
\end{verbatim}
\end{screen}
\end{minipage}
\\
\subsubsection{ファイル形式}
[string01]~[string02]
\subsubsection{パラメータ}
 \begin{itemize}
   \item  $[$string01$]$
   
   {\bf 形式 :} string型 (固定)
   
   {\bf 説明 :} キーワードを指定します。
   
   \item  $[$string02$]$
   
    {\bf 形式 :} string型 

   {\bf 説明 :} キーワードにひも付けられるファイル名を指定します(任意)。
 \end{itemize}
\subsubsection{使用ルール}
本ファイルを使用するにあたってのルールは以下の通りです。
\begin{itemize}
\item キーワードを記載後、半角空白を開けた後にファイル名を書きます。ファイル名は自由に設定できます。
\item ファイル読込用キーワードはTable\ref{Table:Defs}により指定します。
\item 必ず指定しなければいけないキーワードはModPara, LocSpin, Orbitalです。それ以外のキーワードについては、指定がない場合はデフォルト値が採用されます(変分パラメータについては最適化されず、固定する設定となります)。詳細は各ファイルの説明を参照してください。
\item 各キーワードは順不同に記述できます。
\item 指定したキーワード、ファイルが存在しない場合はエラー終了します。
\item $\#$で始まる行は読み飛ばされます。
\end{itemize}

 \begin{table*}[h!]
\begin{center}
  \begin{tabular}{|ll|} \hline
           Keywords     & Details for corresponding files       \\   \hline\hline
           *ModPara       &  Parameters for calculation.        \\ \hline 
           %%%%%%%%%%%%%%%%%%  
           *LocSpin         &  Configurations of the local spins for Hamiltonian.         \\ 
           Trans       &   Transfer and chemical potential for Hamiltonian.  \\
           InterAll  &   Two-body interactions for Hamiltonian. \\  
           CoulombIntra  &   CoulombIntra interactions. \\  
           CoulombInter  &   CoulombInter  interactions. \\  
           Hund  &   Hund couplings. \\  
           PairHop  &  Pair hopping couplings. \\  
           Exchange  &  Exchange couplings. \\  \hline
           %%%%%%%%%%%%%%%%%%
           Gutzwiller & Gutzwiller factors.\\
           Jastrow & Charge Jastrow factors.\\
           DH2 & 2-site doublon-holon correlation factors.\\
           DH4 & 4-site doublon-holon correlation factors.\\
           *Orbital & pair orbital factors.\\
           TransSym & translational and lattice symmetry operation. \\ \hline
           %%%%%%%%%%%%%%%%%%
           InGutzwiller & Initial values of Gutzwiller factors.\\
           InJastrow & Initial values of charge Jastrow factors.\\
           InDH2 & Initial values of 2-site doublon-holon correlation factors.\\
           InDH4 & Initial values of 4-site doublon-holon correlation factors.\\
           InOrbital & Initial values of pair orbital factors.\\ \hline
           %%%%%%%%%%%%%%%%%%
           OneBodyG         &   Output components for Green functions $\langle c_{i\sigma}^{\dagger}c_{j\sigma}\rangle$           \\   
           TwoBodyG &   Output components for Correlation functions $\langle c_{i\sigma}^{\dagger}c_{j\sigma}c_{k\tau}^{\dagger}c_{l\tau}\rangle$  \\   \hline
  \end{tabular}
\end{center}
\caption{List of the definition files. * が付いているファイルは実行時に必ず必要となります。}
\label{Table:Defs}
\end{table*}%

\newpage
%----------------------------------
\subsection{ModParaファイル}
\label{Subsec:modpara}
計算で使用するパラメータを指定します。以下のようなフォーマットをしています。\\
\begin{minipage}{10cm}
\begin{screen}
\begin{verbatim}
--------------------
Model_Parameters   0
--------------------
VMC_Cal_Parameters
--------------------
CDataFileHead  zvo
CParaFileHead  zqp
--------------------
NVMCCalMode    0
NLanczosMode   0
--------------------
NDataIdxStart  1
NDataQtySmp    5
--------------------
Nsite          16
Nelectron      8
NSPGaussLeg    1
NSPStot        0
NMPTrans       1
NSROptItrStep  1200
NSROptItrSmp   100
DSROptRedCut   0.001
DSROptStaDel   0.02
DSROptStepDt   0.02
NVMCWarmUp     10
NVMCIniterval  1
NVMCSample     1000
NExUpdatePath  0
RndSeed        11272
NSplitSize     1
NStore         1  
\end{verbatim}
\end{screen}
\end{minipage}

\subsubsection{ファイル形式}
以下のように行数に応じ異なる形式をとります。
 \begin{itemize}
   \item  1 - 5行:  ヘッダ(何が書かれても問題ありません)。
   \item  6行:  [string01]~[string02]
   \item  7 - 8行: ヘッダ(何が書かれても問題ありません)
   \item  9行以降: [string01]~[int01]
  \end{itemize}
各項目の対応関係は以下の通りです。
\begin{itemize}
   \item  $[$string01$]$
   
   {\bf 形式 :} string型 (固定)

  {\bf 説明 :} キーワードの指定を行います。
   
   \item  $[$string02$]$
   
   {\bf 形式 :} string型 (空白不可)

  {\bf 説明 :} アウトプットファイルのヘッダを指定します。

   \item  $[$int01$]$
   
   {\bf 形式 :} int型 (空白不可)

  {\bf 説明 :} キーワードでひも付けられるパラメータを指定します。

  \end{itemize}

\subsubsection{使用ルール}
本ファイルを使用するにあたってのルールは以下の通りです。
\begin{itemize}
\item 9行目以降ではキーワードを記載後、半角空白を開けた後に整数値を書きます。
\item 行数固定で読み込みを行う為、パラメータの省略はできません。
\end{itemize}

~\subsubsection{キーワード}
 \begin{itemize}
  \item  \verb|CDataFileHead|

 {\bf 形式 :} string型 (空白不可、必須)

{\bf 説明 :} アウトプットファイルのヘッダ。例えば、一体のGreen関数の出力ファイル名が{\bf xxx\_cisajs.dat}として出力されます(xxxに\verb|CDataFileHead|で指定した文字が記載)。

 \item  \verb|CParaFileHead|

 {\bf 形式 :} string型 (空白不可、必須)

{\bf 説明 :} 最適化された変分パラメータの出力ファイル名のヘッダ。最適化された変分パラメータが{\bf xxx\_opt.dat}ファイルとして出力されます(xxxに\verb|CParaFileHead|で指定した文字が記載)。
 
 
 \item  \verb|NVMCCalMode|

 {\bf 形式 :} int型 (\tr{デフォルト値 = 0})

{\bf 説明 :} [0] 変分パラメータの最適化、[1] 1 体・2 体のグリーン関数の計算。
 
 \item  \verb|NLanczosMode|

 {\bf 形式 :} int型 (\tr{デフォルト値 = 0})

{\bf 説明 :} [0] 何もしない、[1] Single Lanczos Step でエネルギーまで計算、[2] Single Lanczos Step で1 体・2 体のグリーン関数まで計算(条件: 1, 2 は\verb|NVMCCalMode| = 1のみ使用可能. また, pair hopping項, exchange 項がハミルトニアンに含まれる場合は使用できません)。
 
 \item  \verb|NDataIdxStart|

 {\bf 形式 :} int型 (\tr{デフォルト値 = 0})

{\bf 説明 :} 出力ファイルの付加番号。\verb|NVMCCalMode|= 0 の場合は\verb|NDataIdxStart|が出力され、 \verb|NVMCCalMode| = 1 の場合は、\verb|NDataIdxStart|から連番で\verb|NDataQtySmp|個のファイルを出力します。
   
 \item  \verb|NDataQtySmp|

 {\bf 形式 :} int型 (\tr{デフォルト値 = 1})

{\bf 説明 :} 出力ファイルのセット数。 \verb|NVMCCalMode| = 1 の場合に使用します。

 \item  \verb|Nsite|

{\bf 形式 :} int型 (1以上、必須)

{\bf 説明 :} サイト数を指定する整数。  

\item  \verb|Nelectron|

{\bf 形式 :} {int型 (1以上、必須)}

{\bf 説明 :} \tr{$\uparrow$電子の個数。$S_z=0$部分空間で計算するので、$\uparrow$電子と$\downarrow$電子の個数は等しい。}

 \item  \verb|NSPGaussLeg|

{\bf 形式 :} {int型 (1以上、\tr{デフォルト値 = 1})}

{\bf 説明 :} スピン量子数射影の$\beta$積分($S^y$回転)のGauss-Legendre求積法の分点数。

 \item  \verb|NSPStot|

{\bf 形式 :} int型 (0以上、\tr{デフォルト値 = 0})

{\bf 説明 :}  スピン量子数。

 \item  \verb|NMPTrans|

{\bf 形式 :} int型 (1以上、\tr{デフォルト値 = 1})

{\bf 説明 :} 
運動量・格子対称性の量子数射影の個数。TransSymファイルで指定した重みで上から\verb|NMPTrans|個まで使用する。\tr{射影を行わない場合は1に設定する。}

 \item  \verb|NSROptItrStep|

{\bf 形式 :} int型 (1以上、\tr{デフォルト値 = 1000})

{\bf 説明 :} 
SR 法で最適化する場合の全ステップ数。\verb|NVMCCalMode|=0の場合のみ使用されます。
 
 \item  \verb|NSROptItrSmp|

{\bf 形式 :} int型 (1以上数、\tr{デフォルト値 = }\verb|NSROptItrStep|/10)

{\bf 説明 :} \verb|NSROptItrStep|ステップ中、最後の\verb|NSROptItrSmp|ステップでの各変分パラメータの平均値を最適値とする。\verb|NVMCCalMode|=0の場合のみ使用されます。

\item   \verb|DSROptRedCut|
   
{\bf 形式 :} double型 (\tr{デフォルト値 = 0.001})

{\bf 説明 :} SR 法安定化因子。手法論文の$\varepsilon_{\rm wf}$に対応。

 \item  \verb|DSROptStaDel| 
   
 {\bf 形式 :} double型 (\tr{デフォルト値 = 0.02})

  {\bf 説明 :} SR 法安定化因子。手法論文の$\varepsilon$に対応。
     
\item \verb|DSROptStepDt|

{\bf 形式 :} double型 

{\bf 説明 :} \tr{要確認(マニュアルに説明なし)}
 
\item \verb|NVMCWarmUp|

{\bf 形式 :} int型 (1以上、\tr{デフォルト値=10})

{\bf 説明 :}マルコフ連鎖の空回し回数。

\item \verb|NVMCIniterval|

{\bf 形式 :} int型 (1以上、\tr{デフォルト値=1})

{\bf 説明 :} サンプル間のステップ間隔。ローカル更新が\verb|Nsite|× \verb|NVMCIniterval| 回行わます。

\item \verb|NVMCSample|

{\bf 形式 :} int型 (1以上、\tr{デフォルト値=1000})

{\bf 説明 :} 期待値計算に使用するサンプル数。

\item \verb|NExUpdatePath|

{\bf 形式 :} int型 (1以上)

{\bf 説明 :} ローカル更新で2 電子交換を[0] 認めない、[1] 認める。\tr{デフォルト設定は局在スピンが一つでもある場合は1、それ以外は0となります}。

\item \verb|RndSeed|

{\bf 形式 :} int型 

{\bf 説明 :} 乱数の初期seed。\tr{MPI 並列では各計算機に}\verb|RndSeed|\tr{+my rank+1 で初期seed が与えられます。}

 \item \verb|NSplitSize|

{\bf 形式 :} int型 (1以上、\tr{デフォルト値=1})

{\bf 説明 :} mpi内部並列を行う場合の並列数。

\item \verb|NStoreOO|

{\bf 形式 :} int型 (1以上、\tr{デフォルト値=1})

{\bf 説明 :} 期待値$\langle O_k O_l \rangle$を計算するとき行列-行列積にするオプション(1で機能On)。
 
 \end{itemize}


\newpage
%----------------------------------
\subsection{LocSpin指定ファイル}
\label{Subsec:locspn}
局在スピンを指定します。以下のようなフォーマットをしています。\\
\begin{minipage}{10cm}
\begin{screen}
\begin{verbatim}
================================ 
NlocalSpin     6  
================================ 
========i_0LocSpn_1IteElc ====== 
================================ 
    0      1
    1      0
    2      1
    3      0
    4      1
    5      0
    6      1
    7      0
    8      1
    9      0
   10      1
   11      0
\end{verbatim}
\end{screen}
\end{minipage}


\subsubsection{ファイル形式}
以下のように行数に応じ異なる形式をとります。
 \begin{itemize}
   \item  1行:  ヘッダ(何が書かれても問題ありません)。
   \item  2行:   [string01]~[int01]
   \item  3-5行:  ヘッダ(何が書かれても問題ありません)。
   \item  6行以降:  [int02]~[int03]
  \end{itemize}
 \subsubsection{パラメータ}
 \begin{itemize}

 \item  $[$string01$]$

 {\bf 形式 :} string型 (空白不可)

{\bf 説明 :} 局在スピンの総数を示すキーワード(任意)。


  \item  $[$int01$]$

 {\bf 形式 :} int型 (空白不可)

{\bf 説明 :} 局在スピンの総数を指定する整数。

 
  \item  $[$int02$]$

 {\bf 形式 :} int型 (空白不可)

{\bf 説明 :} サイト番号を指定する整数。0以上\verb|Nsite|{未満}で指定します。

 
  \item  $[$int03$]$

 {\bf 形式 :} int型 (空白不可)

{\bf 説明 :} 局在スピンか遍歴電子かを指定する整数。\\
{
0: 遍歴電子\\
$n>0$: $2S=n$の局在スピン\\
}
を選択することが出来ます。
 \end{itemize}

\subsubsection{使用ルール}
本ファイルを使用するにあたってのルールは以下の通りです。
\begin{itemize}
\item 行数固定で読み込みを行う為、ヘッダの省略はできません。
\item $[$int01$]$と$[$int03$]$で指定される局在電子数の総数が異なる場合はエラー終了します。
\item $[$int02$]$の総数が全サイト数と異なる場合はエラー終了します。
\item $[$int02$]$が全サイト数以上もしくは負の値をとる場合はエラー終了します。
\end{itemize}


\newpage
\subsection{Trans指定ファイル}
\label{Subsec:Trans}
ここではハミルトニアン
\begin{align}
H +=-\sum_{ij\sigma_1\sigma2} t_{ij\sigma_1\sigma2}c_{i\sigma_1}^{\dag}c_{j\sigma_2}
\end{align}
に対するTransfer積分$t_{ij\sigma_1\sigma2}$を指定します。以下にファイル名を記載します。\\
\begin{minipage}{12.5cm}
\begin{screen}
\begin{verbatim}
======================== 
NTransfer      24  
======================== 
========i_j_s_tijs====== 
======================== 
    0     0     2     0   1.000000  0.000000
    2     0     0     0   1.000000  0.000000
    0     1     2     1   1.000000  0.000000
    2     1     0     1   1.000000  0.000000
    2     0     4     0   1.000000  0.000000
    4     0     2     0   1.000000  0.000000
    2     1     4     1   1.000000  0.000000
    4     1     2     1   1.000000  0.000000
    4     0     6     0   1.000000  0.000000
    6     0     4     0   1.000000  0.000000
    4     1     6     1   1.000000  0.000000
    6     1     4     1   1.000000  0.000000
    6     0     8     0   1.000000  0.000000
    8     0     6     0   1.000000  0.000000
…
\end{verbatim}
\end{screen}
\end{minipage}

\subsubsection{ファイル形式}
以下のように行数に応じ異なる形式をとります。
 \begin{itemize}
   \item  1行:  ヘッダ(何が書かれても問題ありません)。
   \item  2行:   [string01]~[int01]
   \item  3-5行:  ヘッダ(何が書かれても問題ありません)。
   \item  6行以降: [int02]~~[int03]~~[int04]~~[int05]~~[double01]~~[double02] 
  \end{itemize}
\subsubsection{パラメータ}
 \begin{itemize}

   \item  $[$string01$]$
   
    {\bf 形式 :} string型 (空白不可)

   {\bf 説明 :} Transfer総数のキーワード名を指定します(任意)。

   \item  $[$int01$]$
   
    {\bf 形式 :} int型 (空白不可)

   {\bf 説明 :} Transferの総数を指定します。

  \item  $[$int02$]$, $[$int04$]$

 {\bf 形式 :} int型 (空白不可)

{\bf 説明 :} サイト番号を指定する整数。0以上\verb|Nsite|{未満}で指定します。
 
  \item  $[$int03$]$, $[$int05$]$

 {\bf 形式 :} int型 (空白不可)

{\bf 説明 :} スピンを指定する整数。\\
0: アップスピン\\
1: ダウンスピン\\
を選択することが出来ます。


 \item  $[$double01$]$
   
   {\bf 形式 :} double型 (空白不可)

  {\bf 説明 :}  $t_{ij\sigma_1\sigma_2}$の実部を指定します。

 \item  $[$double02$]$
   
   {\bf 形式 :} double型 (空白不可)

  {\bf 説明 :}  $t_{ij\sigma_1\sigma_2}$の虚部を指定します。
\end{itemize}

\subsubsection{使用ルール}
本ファイルを使用するにあたってのルールは以下の通りです。
\begin{itemize}
\item 行数固定で読み込みを行う為、ヘッダの省略はできません。
\item Hamiltonianがエルミートという制限から$t_{ij\sigma_1\sigma_2}=t_{ji\sigma_2\sigma_1}^{\dagger}$の関係を満たす必要があります。上記の関係が成立しない場合にはエラー終了します。
\item 成分が重複して指定された場合にはエラー終了します。
\item $[$int01$]$と定義されているTrasferの総数が異なる場合はエラー終了します。
\item $[$int02$]$-$[$int05$]$を指定する際、範囲外の整数を指定した場合はエラー終了します。
\end{itemize}

\newpage
\subsection{InterAll指定ファイル}
\label{Subsec:interall}
ここでは一般二体相互作用をハミルトニアンに付け加えます。付け加える項は以下で与えられます。
\begin{equation}
H+=\sum_{i,j,k,l}\sum_{\sigma_1,\sigma_2, \sigma_3, \sigma_4}
I_{ijkl\sigma_1\sigma_2\sigma_3\sigma_4}c_{i\sigma_1}^{\dagger}c_{j\sigma_2}c_{k\sigma_3}^{\dagger}c_{l\sigma_4}
\end{equation}
なお、スピンに関して計算する場合には、$i=j, k=l$となるよう設定してください。
以下にファイル例を記載します。

\begin{minipage}{12.5cm}
\begin{screen}
\begin{verbatim}
====================== 
NInterAll      36  
====================== 
========zInterAll===== 
====================== 
0    0    0    1    1    1    1    0   0.50  0.0
0    1    0    0    1    0    1    1   0.50  0.0
0    0    0    0    1    0    1    0   0.25  0.0
0    0    0    0    1    1    1    1  -0.25  0.0
0    1    0    1    1    0    1    0  -0.25  0.0
0    1    0    1    1    1    1    1   0.25  0.0
2    0    2    1    3    1    3    0   0.50  0.0
2    1    2    0    3    0    3    1   0.50  0.0
2    0    2    0    3    0    3    0   0.25  0.0
2    0    2    0    3    1    3    1  -0.25  0.0
2    1    2    1    3    0    3    0  -0.25  0.0
2    1    2    1    3    1    3    1   0.25  0.0
4    0    4    1    5    1    5    0   0.50  0.0
4    1    4    0    5    0    5    1   0.50  0.0
4    0    4    0    5    0    5    0   0.25  0.0
4    0    4    0    5    1    5    1  -0.25  0.0
4    1    4    1    5    0    5    0  -0.25  0.0
4    1    4    1    5    1    5    1   0.25  0.0
…
\end{verbatim}
\end{screen}
\end{minipage}

\subsubsection{ファイル形式}
以下のように行数に応じ異なる形式をとります。
 \begin{itemize}
   \item  1行:  ヘッダ(何が書かれても問題ありません)。
   \item  2行:   [string01]~[int01]
   \item  3-5行:  ヘッダ(何が書かれても問題ありません)。
   \item  6行以降:
   [int02]~[int03]~[int04]~[int05]~[int06]~[int07]~[int08]~[int09]~[double01]~[double02] 
  \end{itemize}
\subsubsection{パラメータ}
 \begin{itemize}

   \item  $[$string01$]$
   
    {\bf 形式 :} string型 (空白不可)

   {\bf 説明 :} 二体相互作用の総数のキーワード名を指定します(任意)。

   \item  $[$int01$]$
   
    {\bf 形式 :} int型 (空白不可)

   {\bf 説明 :} 二体相互作用の総数を指定します。

  \item  $[$int02$]$, $[$int04$]$, $[$int06$]$, $[$int08$]$

 {\bf 形式 :} int型 (空白不可)

{\bf 説明 :} サイト番号を指定する整数。0以上\verb|Nsite|{未満}で指定します。
 
  \item  $[$int03$]$, $[$int05$]$, $[$int07$]$, $[$int09$]$

 {\bf 形式 :} int型 (空白不可)

{\bf 説明 :} スピンを指定する整数。\\
0: アップスピン\\
1: ダウンスピン\\
を選択することが出来ます。


 \item  $[$double01$]$
   
   {\bf 形式 :} double型 (空白不可)

  {\bf 説明 :}  $I_{ijkl\sigma_1\sigma_2\sigma_3\sigma_4}$の実部を指定します。

 \item  $[$double02$]$
   
   {\bf 形式 :} double型 (空白不可)

  {\bf 説明 :}  $I_{ijkl\sigma_1\sigma_2\sigma_3\sigma_4}$の虚部を指定します。
\end{itemize}


\subsubsection{使用ルール}
本ファイルを使用するにあたってのルールは以下の通りです。
\begin{itemize}
\item 行数固定で読み込みを行う為、ヘッダの省略はできません。
\item Hamiltonianがエルミートという制限から$I_{ijkl\sigma_1\sigma_2\sigma_3\sigma_4}=I_{lkji\sigma_4\sigma_3\sigma_2\sigma_1}^{\dag}$の関係を満たす必要があります。上記の関係が成立しない場合にはエラー終了します。
また、エルミート共役の形式は$I_{ijkl\sigma_1\sigma_2\sigma_3\sigma_4}c_{i\sigma_1}^{\dagger}c_{j\sigma_2}c_{k\sigma_3}^{\dagger}c_{l\sigma_4}$に対して、$I_{lkji\sigma_4\sigma_3\sigma_2\sigma_1}$
$c_{l\sigma_4}^{\dagger}c_{k\sigma_3}c_{j\sigma_2}^{\dagger}c_{i\sigma_1}$を満たすように入力してください。
\item スピンに関して計算する場合、$i=j, k=l$を満たさないペアが存在するとエラー終了します。
\item 成分が重複して指定された場合にはエラー終了します。
\item $[$int01$]$と定義されているInterAllの総数が異なる場合はエラー終了します。
\item $[$int02$]$-$[$int09$]$を指定する際、範囲外の整数を指定した場合はエラー終了します。
\end{itemize}


\newpage
\subsection{CoulombIntra指定ファイル}
\label{Subsec:coulombintra}
オンサイトクーロン相互作用をハミルトニアンに付け加えます。付け加える項は以下で与えられます。
\begin{equation}
H+=\sum_{i}U_i n_ {i \uparrow}n_{i \downarrow}
\end{equation}
以下にファイル例を記載します。

\begin{minipage}{12.5cm}
\begin{screen}
\begin{verbatim}
====================== 
NCoulombIntra 6  
====================== 
========i_0LocSpn_1IteElc ====== 
====================== 
   0  4.000000
   1  4.000000
   2  4.000000
   3  4.000000
   4  4.000000
   5  4.000000
\end{verbatim}
\end{screen}
\end{minipage}

\subsubsection{ファイル形式}
以下のように行数に応じ異なる形式をとります。
 \begin{itemize}
   \item  1行:  ヘッダ(何が書かれても問題ありません)。
   \item  2行:   [string01]~[int01]
   \item  3-5行:  ヘッダ(何が書かれても問題ありません)。
   \item  6行以降:
   [int02]~[double01] 
  \end{itemize}
\subsubsection{パラメータ}
 \begin{itemize}

   \item  $[$string01$]$
   
    {\bf 形式 :} string型 (空白不可)

   {\bf 説明 :} オンサイトクーロン相互作用の総数のキーワード名を指定します(任意)。

   \item  $[$int01$]$
   
    {\bf 形式 :} int型 (空白不可)

   {\bf 説明 :} オンサイトクーロン相互作用の総数を指定します。

  \item  $[$int02$]$
  
 {\bf 形式 :} int型 (空白不可)

{\bf 説明 :} サイト番号を指定する整数。0以上\verb|Nsite|{未満}で指定します。
 
 \item  $[$double01$]$
   
   {\bf 形式 :} double型 (空白不可)

  {\bf 説明 :}  $U_i$を指定します。
  
\end{itemize}

\subsubsection{使用ルール}
本ファイルを使用するにあたってのルールは以下の通りです。
\begin{itemize}
\item 行数固定で読み込みを行う為、ヘッダの省略はできません。
\item 成分が重複して指定された場合にはエラー終了します。
\item $[$int01$]$と定義されているオンサイトクーロン相互作用の総数が異なる場合はエラー終了します。
\item $[$int02$]$を指定する際、範囲外の整数を指定した場合はエラー終了します。
\end{itemize}


\newpage
\subsection{CoulombInter指定ファイル}
オフサイトクーロン相互作用をハミルトニアンに付け加えます。付け加える項は以下で与えられます。
\begin{equation}
H+=\sum_{i,j}V_{ij} n_ {i}n_{j}
\end{equation}
以下にファイル例を記載します。

\begin{minipage}{12.5cm}
\begin{screen}
\begin{verbatim}
====================== 
NCoulombInter 6  
====================== 
========CoulombInter ====== 
====================== 
   0     1 -0.125000
   1     2 -0.125000
   2     3 -0.125000
   3     4 -0.125000
   4     5 -0.125000
   5     0 -0.125000
\end{verbatim}
\end{screen}
\end{minipage}

\subsubsection{ファイル形式}
以下のように行数に応じ異なる形式をとります。
 \begin{itemize}
   \item  1行:  ヘッダ(何が書かれても問題ありません)。
   \item  2行:   [string01]~[int01]
   \item  3-5行:  ヘッダ(何が書かれても問題ありません)。
   \item  6行以降:
   [int02]~[int03]~[double01] 
  \end{itemize}
\subsubsection{パラメータ}
 \begin{itemize}

   \item  $[$string01$]$
   
    {\bf 形式 :} string型 (空白不可)

   {\bf 説明 :} オフサイトクーロン相互作用の総数のキーワード名を指定します(任意)。

   \item  $[$int01$]$
   
    {\bf 形式 :} int型 (空白不可)

   {\bf 説明 :} オフサイトクーロン相互作用の総数を指定します。

  \item  $[$int02$]$, $[$int03$]$
  
 {\bf 形式 :} int型 (空白不可)

{\bf 説明 :} サイト番号を指定する整数。0以上\verb|Nsite|{未満}で指定します。
 
 \item  $[$double01$]$
   
   {\bf 形式 :} double型 (空白不可)

  {\bf 説明 :}  $V_{ij}$を指定します。
  
\end{itemize}

\subsubsection{使用ルール}
本ファイルを使用するにあたってのルールは以下の通りです。
\begin{itemize}
\item 行数固定で読み込みを行う為、ヘッダの省略はできません。
\item 成分が重複して指定された場合にはエラー終了します。
\item $[$int01$]$と定義されているオフサイトクーロン相互作用の総数が異なる場合はエラー終了します。
\item $[$int02$]$-$[$int03$]$を指定する際、範囲外の整数を指定した場合はエラー終了します。
\end{itemize}

\newpage
\subsection{Hund指定ファイル}
Hundカップリングをハミルトニアンに付け加えます。付け加える項は以下で与えられます。
\begin{equation}
H+=-\sum_{i,j}J_{ij}^{\rm Hund} (n_{i\uparrow}n_{j\uparrow}+n_{i\downarrow}n_{j\downarrow})
\end{equation}
以下にファイル例を記載します。

\begin{minipage}{12.5cm}
\begin{screen}
\begin{verbatim}
====================== 
NHund 6  
====================== 
========Hund ====== 
====================== 
   0     1 -0.250000
   1     2 -0.250000
   2     3 -0.250000
   3     4 -0.250000
   4     5 -0.250000
   5     0 -0.250000
\end{verbatim}
\end{screen}
\end{minipage}

\subsubsection{ファイル形式}
以下のように行数に応じ異なる形式をとります。
 \begin{itemize}
   \item  1行:  ヘッダ(何が書かれても問題ありません)。
   \item  2行:   [string01]~[int01]
   \item  3-5行:  ヘッダ(何が書かれても問題ありません)。
   \item  6行以降:
   [int02]~[int03]~[double01] 
  \end{itemize}
\subsubsection{パラメータ}
 \begin{itemize}

   \item  $[$string01$]$
   
    {\bf 形式 :} string型 (空白不可)

   {\bf 説明 :} Hundカップリングの総数のキーワード名を指定します(任意)。

   \item  $[$int01$]$
   
    {\bf 形式 :} int型 (空白不可)

   {\bf 説明 :} Hundカップリングの総数を指定します。

  \item  $[$int02$]$, $[$int03$]$
  
 {\bf 形式 :} int型 (空白不可)

{\bf 説明 :} サイト番号を指定する整数。0以上\verb|Nsite|{未満}で指定します。
 
 \item  $[$double01$]$
   
   {\bf 形式 :} double型 (空白不可)

  {\bf 説明 :}  $J_{ij}^{\rm Hund}$を指定します。
  
\end{itemize}

\subsubsection{使用ルール}
本ファイルを使用するにあたってのルールは以下の通りです。
\begin{itemize}
\item 行数固定で読み込みを行う為、ヘッダの省略はできません。
\item 成分が重複して指定された場合にはエラー終了します。
\item $[$int01$]$と定義されているHundカップリングの総数が異なる場合はエラー終了します。
\item $[$int02$]$-$[$int03$]$を指定する際、範囲外の整数を指定した場合はエラー終了します。
\end{itemize}

\newpage
\subsection{PairHop指定ファイル}
PairHopカップリングをハミルトニアンに付け加えます。付け加える項は以下で与えられます。
\begin{equation}
H+=\sum_{i,j}J_{ij}^{\rm Pair} c_ {i \uparrow}^{\dag}c_{j\uparrow}c_{i \downarrow}^{\dag}c_{j  \downarrow}
\end{equation}
以下にファイル例を記載します。

\begin{minipage}{12.5cm}
\begin{screen}
\begin{verbatim}
====================== 
NPairhop 12  
====================== 
========Pairhop ====== 
====================== 
   0     1  0.50000
   1     0  0.50000  
   1     2  0.50000
   2     1  0.50000
   2     3  0.50000
   3     2  0.50000
   3     4  0.50000
   4     3  0.50000
   4     5  0.50000
   5     4  0.50000
   5     0  0.50000
   0     5  0.50000
\end{verbatim}
\end{screen}
\end{minipage}

\subsubsection{ファイル形式}
以下のように行数に応じ異なる形式をとります。
 \begin{itemize}
   \item  1行:  ヘッダ(何が書かれても問題ありません)。
   \item  2行:   [string01]~[int01]
   \item  3-5行:  ヘッダ(何が書かれても問題ありません)。
   \item  6行以降:
   [int02]~[int03]~[double01] 
  \end{itemize}
\subsubsection{パラメータ}
 \begin{itemize}

   \item  $[$string01$]$
   
    {\bf 形式 :} string型 (空白不可)

   {\bf 説明 :} PairHopカップリングの総数のキーワード名を指定します(任意)。

   \item  $[$int01$]$
   
    {\bf 形式 :} int型 (空白不可)

   {\bf 説明 :} PairHopカップリングの総数を指定します。

  \item  $[$int02$]$, $[$int03$]$
  
 {\bf 形式 :} int型 (空白不可)

{\bf 説明 :} サイト番号を指定する整数。0以上\verb|Nsite|{未満}で指定します。
 
 \item  $[$double01$]$
   
   {\bf 形式 :} double型 (空白不可)

  {\bf 説明 :}  $J_{ij}^{\rm Pair}$を指定します。
  
\end{itemize}

\subsubsection{使用ルール}
本ファイルを使用するにあたってのルールは以下の通りです。
\begin{itemize}
\item 行数固定で読み込みを行う為、ヘッダの省略はできません。
\item 成分が重複して指定された場合にはエラー終了します。
\item $[$int01$]$と定義されているPairHopカップリングの総数が異なる場合はエラー終了します。
\item $[$int02$]$-$[$int03$]$を指定する際、範囲外の整数を指定した場合はエラー終了します。
\end{itemize}

\newpage
\subsection{Exchange指定ファイル}
Exchangeカップリングをハミルトニアンに付け加えます。
電子系の場合には
\begin{equation}
H+=\sum_{i,j}J_{ij}^{\rm Ex} (c_ {i \uparrow}^{\dag}c_{j\uparrow}c_{j \downarrow}^{\dag}c_{i  \downarrow}+c_ {i \downarrow}^{\dag}c_{j\downarrow}c_{j \uparrow}^{\dag}c_{i  \uparrow})
\end{equation}
が付け加えられ、スピン系の場合には
\begin{equation}
H+=\sum_{i,j}J_{ij}^{\rm Ex} (S_i^+S_j^-+S_i^-S_j^+)
\end{equation}
が付け加えられます。
{\bf スピン系の$(S_i^+S_j^-+S_i^-S_j^+)$を
電子系の演算子で書き直すと、
$-(c_ {i \uparrow}^{\dag}c_{j\uparrow}c_{j \downarrow}^{\dag}c_{i  \downarrow}+c_ {i \downarrow}^{\dag}c_{j\downarrow}c_{j \uparrow}^{\dag}c_{i  \uparrow})$
となることに注意して下さい。}
以下にファイル例を記載します。

\begin{minipage}{12.5cm}
\begin{screen}
\begin{verbatim}
====================== 
NExchange 6  
====================== 
========Exchange ====== 
====================== 
   0     1  0.50000
   1     2  0.50000
   2     3  0.50000
   3     4  0.50000
   4     5  0.50000
   5     0  0.50000
\end{verbatim}
\end{screen}
\end{minipage}

\subsubsection{ファイル形式}
以下のように行数に応じ異なる形式をとります。
 \begin{itemize}
   \item  1行:  ヘッダ(何が書かれても問題ありません)。
   \item  2行:   [string01]~[int01]
   \item  3-5行:  ヘッダ(何が書かれても問題ありません)。
   \item  6行以降:
   [int02]~[int03]~[double01] 
  \end{itemize}
\subsubsection{パラメータ}
 \begin{itemize}

   \item  $[$string01$]$
   
    {\bf 形式 :} string型 (空白不可)

   {\bf 説明 :} Exchangeカップリングの総数のキーワード名を指定します(任意)。

   \item  $[$int01$]$
   
    {\bf 形式 :} int型 (空白不可)

   {\bf 説明 :} Exchangeカップリングの総数を指定します。

  \item  $[$int02$]$, $[$int03$]$
  
 {\bf 形式 :} int型 (空白不可)

{\bf 説明 :} サイト番号を指定する整数。0以上\verb|Nsite|{未満}で指定します。
 
 \item  $[$double01$]$
   
   {\bf 形式 :} double型 (空白不可)

  {\bf 説明 :}  $J_{ij}^{\rm Ex}$を指定します。
  
\end{itemize}

\subsubsection{使用ルール}
本ファイルを使用するにあたってのルールは以下の通りです。
\begin{itemize}
\item 行数固定で読み込みを行う為、ヘッダの省略はできません。
\item 成分が重複して指定された場合にはエラー終了します。
\item $[$int01$]$と定義されているExchangeカップリングの総数が異なる場合はエラー終了します。
\item $[$int02$]$-$[$int03$]$を指定する際、範囲外の整数を指定した場合はエラー終了します。
\end{itemize}

\newpage
\subsection{Gutzwiller指定ファイル}
\label{Subsec:Gutzwiller}

Gutzwiller因子
\begin{equation}
{\cal P}_G=\exp\left[ \sum_i g_i n_{i\uparrow} n_{i\downarrow} \right]
\end{equation}
の設定を行います。指定するパラメータはサイト番号$i$と$g_i$の変分パラメータの番号です。
以下にファイル例を記載します。

\begin{minipage}{12.5cm}
\begin{screen}
\begin{verbatim}
======================
NGutzwillerIdx 2  
ComplexType 0
====================== 
====================== 
   0     0
   1     0
   2     0
   3     1
(continue...)
  12     1
  13     0
  14     0
  15     0
   0     1
   1     0
\end{verbatim}
\end{screen}
\end{minipage}

\subsubsection{ファイル形式}
以下のように行数に応じ異なる形式をとります($N_s$はサイト数、$N_g$は変分パラメータの種類の数)。
 \begin{itemize}
   \item  1行:  ヘッダ(何が書かれても問題ありません)。
   \item  2行:   [string01]~[int01]
   \item  3行:   [string02]~[int02]
   \item  4-5行:  ヘッダ(何が書かれても問題ありません)。
   \item  6 - 5+$N_s$行: [int03]~[int04]
   \item  6+$N_s$ - 5+$N_s$+$N_g$行:[int05]~[int06]
  \end{itemize}
\subsubsection{パラメータ}
 \begin{itemize}

   \item  $[$string01$]$
   
    {\bf 形式 :} string型 (空白不可)

   {\bf 説明 :} $g_i$の変分パラメータの種類の総数のキーワード名を指定します(任意)。

   \item  $[$int01$]$
   
    {\bf 形式 :} int型 (空白不可)

   {\bf 説明 :} $g_i$の変分パラメータの種類の総数を指定します。

   \item  $[$string02$]$
   
    {\bf 形式 :} string型 (空白不可)

   {\bf 説明 :} $g_i$の変分パラメータの型を指定するためのキーワード名を指定します(任意)。

  \item  $[$int02$]$
  
 {\bf 形式 :} int型 (空白不可)

{\bf 説明 :} 変分パラメータの型を指定する整数。0が実数、1が複素数に対応します。

  \item  $[$int03$]$
  
 {\bf 形式 :} int型 (空白不可)

{\bf 説明 :} サイト番号を指定する整数。0以上\verb|Nsite|{未満}で指定します。
 
 \item  $[$int04$]$
   
   {\bf 形式 :} int型 (空白不可)

  {\bf 説明 :} $g_i$の変分パラメータの種類を表します。0以上[int01]{未満}で指定します。

 \item  $[$int05$]$
   
   {\bf 形式 :} int型 (空白不可)

  {\bf 説明 :} $g_i$の変分パラメータの種類を表します(最適化有無の設定用)。0以上[int01]{未満}で指定します。

 \item  $[$int06$]$
   
   {\bf 形式 :} int型 (空白不可)

  {\bf 説明 :} [int05]で指定した$g_i$の変分パラメータの最適化有無を設定します。最適化する場合は1、最適化しない場合は0とします。
  
\end{itemize}

\subsubsection{使用ルール}
本ファイルを使用するにあたってのルールは以下の通りです。
\begin{itemize}
\item 行数固定で読み込みを行う為、ヘッダの省略はできません。
\item 成分が重複して指定された場合にはエラー終了します。
\item $[$int01$]$と定義されている変分パラメータの種類の総数が異なる場合はエラー終了します。
\item $[$int02$]$-$[$int06$]$を指定する際、範囲外の整数を指定した場合はエラー終了します。
\end{itemize}

\newpage
\subsection{Jastrow指定ファイル}
\label{Subsec:Jastrow}
Jastrow因子
\begin{equation}
{\cal P}_J=\exp\left[\frac{1}{2} \sum_{i\neq j} v_{ij} n_i n_j\right]
\end{equation}
の設定を行います。指定するパラメータはサイト番号$i, j$と$v_{ij}$の変分パラメータの番号です。以下にファイル例を記載します。

\begin{minipage}{12.5cm}
\begin{screen}
\begin{verbatim}
======================
NJastrowIdx 5  
ComplexType 0
====================== 
======================
   0     1     0 
   0     2     1 
   0     3     0 
 (continue...)
   0    1 
   1    1 
   2    1 
   3    1 
   4    1 
\end{verbatim}
\end{screen}
\end{minipage}

\subsubsection{ファイル形式}
以下のように行数に応じ異なる形式をとります($N_s$はサイト数、$N_j$は変分パラメータの種類の数)。
 \begin{itemize}
   \item  1行:  ヘッダ(何が書かれても問題ありません)。
   \item  2行:   [string01]~[int01]
   \item  3行:   [string02]~[int02]
   \item  4-5行:  ヘッダ(何が書かれても問題ありません)。
   \item  6 - 5+$N_s\times (N_s-1)$行: [int03]~[int04]~[int05]
   \item  6+$N_s\times (N_s-1)$ - 5+$N_s\times (N_s-1)$+$N_j$行:[int06]~[int07]
  \end{itemize}
\subsubsection{パラメータ}
 \begin{itemize}

   \item  $[$string01$]$
   
    {\bf 形式 :} string型 (空白不可)

   {\bf 説明 :} $v_{ij}$の変分パラメータの種類の総数のキーワード名を指定します(任意)。

   \item  $[$int01$]$
   
    {\bf 形式 :} int型 (空白不可)

   {\bf 説明 :} $v_{ij}$の変分パラメータの種類の総数を指定します。

   \item  $[$string02$]$
   
    {\bf 形式 :} string型 (空白不可)

   {\bf 説明 :} $v_{ij}$の変分パラメータの型を指定するためのキーワード名を指定します(任意)。

   \item  $[$int02$]$
   
    {\bf 形式 :} int型 (空白不可)

   {\bf 説明 :} $v_{ij}$の変分パラメータの型を指定します。0が実数、1が複素数に対応します。

  \item  $[$int03$]$, $[$int04$]$
  
 {\bf 形式 :} int型 (空白不可)

{\bf 説明 :} サイト番号を指定する整数。0以上\verb|Nsite|{未満}で指定します。
 
 \item  $[$int05$]$
   
   {\bf 形式 :} int型 (空白不可)

  {\bf 説明 :} $v_{ij}$の変分パラメータの種類を表します。0以上[int01]{未満}で指定します。

 \item  $[$int06$]$
   
   {\bf 形式 :} int型 (空白不可)

  {\bf 説明 :} $v_{ij}$の変分パラメータの種類を表します(最適化有無の設定用)。0以上[int01]{未満}で指定します。

 \item  $[$int07$]$
   
   {\bf 形式 :} int型 (空白不可)

  {\bf 説明 :} [int06]で指定した$v_{ij}$の変分パラメータの最適化有無を設定します。最適化する場合は1、最適化しない場合は0とします。
  
\end{itemize}

\subsubsection{使用ルール}
本ファイルを使用するにあたってのルールは以下の通りです。
\begin{itemize}
\item 行数固定で読み込みを行う為、ヘッダの省略はできません。
\item 成分が重複して指定された場合にはエラー終了します。
\item $[$int01$]$と定義されている変分パラメータの種類の総数が異なる場合はエラー終了します。
\item $[$int02$]$-$[$int07$]$を指定する際、範囲外の整数を指定した場合はエラー終了します。
\end{itemize}

\newpage
\subsection{DH2指定ファイル}
\label{Subsec:DH2}

\begin{equation}
{\cal P}_{d-h}^{(2)}= \exp \left[ \sum_t \sum_{n=0}^2 (\alpha_{2nt}^d \sum_{i}\xi_{i2nt}^d+\alpha_{2nt}^h \sum_{i}\xi_{i2nt}^h)\right]
\end{equation}
で表される2サイトのdoublon-holon相関因子の設定を行います。指定するパラメータはサイト番号$i$とその周囲2サイト、$\alpha_{2nt}^{d,h}$の変分パラメータの番号で、変分パラメータは各サイト毎に$t$種類設定します。以下にファイル例を記載します。

\begin{minipage}{12.5cm}
\begin{screen}
\begin{verbatim}
====================================
NDoublonHolon2siteIdx 2  
ComplexType 0
====================================
====================================
   0     5   15    0
   0    13    7    1
   1     6   12    0
   1    14    4    1
 (continue...)
  15     0   10    0
  15     8    2    1
   0     1 
   1     1 
   2     1 
(continue...)
  10     1 
  11     1 
\end{verbatim}
\end{screen}
\end{minipage}

\subsubsection{ファイル形式}
以下のように行数に応じ異なる形式をとります($N_s$はサイト数、$N_{\rm dh2}$は変分パラメータの種類の数)。
 \begin{itemize}
   \item  1行:  ヘッダ(何が書かれても問題ありません)。
   \item  2行:   [string01]~[int01]
   \item  3行:   [string02]~[int02]
   \item  4-5行:  ヘッダ(何が書かれても問題ありません)。
   \item  6 - 5+$N_s\times N_{\rm dh2}$行: [int03]~[int04]~[int05]~[int06]
   \item  6+$N_s\times N_{\rm dh2}$ - 5+$(N_s+6) \times N_{\rm dh2}$行:[int07]~[int08]
  \end{itemize}
\subsubsection{パラメータ}
 \begin{itemize}

   \item  $[$string01$]$
   
    {\bf 形式 :} string型 (空白不可)

   {\bf 説明 :} 変分パラメータのセット総数のキーワード名を指定します(任意)。

   \item  $[$int01$]$
   
    {\bf 形式 :} int型 (空白不可)

   {\bf 説明 :} 変分パラメータのセット総数を指定します。

  \item  $[$string02$]$
   
    {\bf 形式 :} string型 (空白不可)

   {\bf 説明 :}変分パラメータの型を指定するためのキーワード名を指定します(任意)。
   
   \item  $[$int02$]$
   
    {\bf 形式 :} int型 (空白不可)

   {\bf 説明 :} 変分パラメータの型を指定します。0が実数、1が複素数に対応します。

  \item  $[$int03$]$,  $[$int04$]$, $[$int05$]$
   
 {\bf 形式 :} int型 (空白不可)

{\bf 説明 :} サイト番号を指定する整数。0以上\verb|Nsite|{未満}で指定します。
 
 \item  $[$int06$]$
   
   {\bf 形式 :} int型 (空白不可)

  {\bf 説明 :} 変分パラメータの種類を表します。0以上[int01]{未満}で指定します。

 \item  $[$int07$]$
   
   {\bf 形式 :} int型 (空白不可)

  {\bf 説明 :} 変分パラメータの種類を表します(最適化有無の設定用)。値は
  \begin{itemize}
  \item{$n$}: 周囲のdoublon(holon)数 (0, 1, 2)  \\
  \item{$s$}: 中心がdoublonの場合$0$, 中心がholonの場合$1$ \\
  \item{$t$}: 変分パラメータのセット番号(0, $\cdots$ [int1]-1)
  \end{itemize}  
  として、$(2n+s)\times$[int01]$+t$を設定します。
  
 \item  $[$int08$]$
   
   {\bf 形式 :} int型 (空白不可)

  {\bf 説明 :} [int07]で指定した変分パラメータの最適化有無を設定します。最適化する場合は1、最適化しない場合は0とします。
  
  
\end{itemize}

\subsubsection{使用ルール}
本ファイルを使用するにあたってのルールは以下の通りです。
\begin{itemize}
\item 行数固定で読み込みを行う為、ヘッダの省略はできません。
\item 成分が重複して指定された場合にはエラー終了します。
\item $[$int01$]$と定義されている変分パラメータの種類の総数が異なる場合はエラー終了します。
\item \tr{$[$int02$]$-$[$int08$]$を指定する際、範囲外の整数を指定した場合はエラー終了します。}
\end{itemize}


\newpage
\subsection{DH4指定ファイル}
\label{Subsec:DH4}

\begin{equation}
{\cal P}_{d-h}^{(4)}= \exp \left[ \sum_t \sum_{n=0}^4 (\alpha_{4nt}^d \sum_{i}\xi_{i4nt}^d+\alpha_{4nt}^h \sum_{i}\xi_{i4nt}^h)\right]
\end{equation}
で表される4サイトのdoublon-holon相関因子の設定を行います。指定するパラメータはサイト番号$i$とその周囲4サイト、$\alpha_{4nt}^{d,h}$の変分パラメータの番号で、変分パラメータは各サイト毎に$t$種類設定します。以下にファイル例を記載します。

\begin{minipage}{12.5cm}
\begin{screen}
\begin{verbatim}
====================================
NDoublonHolon4siteIdx 1  
ComplexType 0
====================================
====================================
   0     1    3    4   12    0
   1     2    0    5   13    0
   2     3    1    6   14    0
   3     0    2    7   15    0
 (continue...)
  14    15   13    2   10    0
  15    12   14    3   11    0
   0     1 
   1     1 
(continue...)
   8     1 
   9     1 
\end{verbatim}
\end{screen}
\end{minipage}

\subsubsection{ファイル形式}
以下のように行数に応じ異なる形式をとります($N_s$はサイト数、$N_{\rm dh4}$は変分パラメータの種類の数)。
 \begin{itemize}
   \item  1行:  ヘッダ(何が書かれても問題ありません)。
   \item  2行:   [string01]~[int01]
   \item  3行:   [string02]~[int02]
   \item  4-5行:  ヘッダ(何が書かれても問題ありません)。
   \item  6 - 5+$N_s\times N_{\rm dh4}$行: [int03]~[int04]~[int05]~[int06]~[int07]~[int08]
   \item  6+$N_s\times N_{\rm dh4}$ - 5+$(N_s+10) \times N_{\rm dh4}$行:[int09]~[int10]
  \end{itemize}
\subsubsection{パラメータ}
 \begin{itemize}

   \item  $[$string01$]$
   
    {\bf 形式 :} string型 (空白不可)

   {\bf 説明 :} 変分パラメータのセット総数のキーワード名を指定します(任意)。

   \item  $[$int01$]$
   
    {\bf 形式 :} int型 (空白不可)

   {\bf 説明 :} 変分パラメータのセット総数を指定します。

   \item  $[$string02$]$
   
    {\bf 形式 :} string型 (空白不可)

   {\bf 説明 :} 変分パラメータの型を指定するためのキーワード名を指定します(任意)。

   \item  $[$int02$]$
   
    {\bf 形式 :} int型 (空白不可)

   {\bf 説明 :} 変分パラメータの型を指定します。0が実数、1が複素数に対応します。

  \item   $[$int03$]$,  $[$int04$]$, $[$int05$]$, $[$int06$]$, $[$int07$]$
   
 {\bf 形式 :} int型 (空白不可)

{\bf 説明 :} サイト番号を指定する整数。0以上\verb|Nsite|{未満}で指定します。
 
 \item  $[$int08$]$
   
   {\bf 形式 :} int型 (空白不可)

  {\bf 説明 :} 変分パラメータの種類を表します。0以上[int01]{未満}で指定します。

 \item  $[$int09$]$
   
   {\bf 形式 :} int型 (空白不可)

  {\bf 説明 :} 変分パラメータの種類を表します(最適化有無の設定用)。値は
  \begin{itemize}
  \item{$n$}: 周囲のdoublon(holon)数 (0, 1, 2, 3, 4)  \\
  \item{$s$}: 中心がdoublonの場合$0$, 中心がholonの場合$1$ \\
  \item{$t$}: 変分パラメータのセット番号(0, $\cdots$ [int1]-1)
  \end{itemize}  
  として、$(2n+s)\times$[int01]$+t$を設定します。
  
 \item  $[$int10$]$
   
   {\bf 形式 :} int型 (空白不可)

  {\bf 説明 :} [int09]で指定した変分パラメータの最適化有無を設定します。最適化する場合は1、最適化しない場合は0とします。
  
\end{itemize}

\subsubsection{使用ルール}
本ファイルを使用するにあたってのルールは以下の通りです。
\begin{itemize}
\item 行数固定で読み込みを行う為、ヘッダの省略はできません。
\item 成分が重複して指定された場合にはエラー終了します。
\item $[$int01$]$と定義されている変分パラメータの種類の総数が異なる場合はエラー終了します。
\item \tr{$[$int02$]$-$[$int10$]$を指定する際、範囲外の整数を指定した場合はエラー終了します。}
\end{itemize}

\newpage
\subsection{Orbital指定ファイル}
\label{Subsec:Orbital}

\begin{equation}
|\phi_{\rm pair} \rangle = \left[\sum_{i, j=1}^{N_s} f_{ij}c_{i\uparrow}^{\dag}c_{j\downarrow}^{\dag} \right]^{N/2}|0 \rangle
\end{equation}
で表されるペア軌道の設定を行います。指定するパラメータはサイト番号$i, j$と変分パラメータの種類を設定します。以下にファイル例を記載します。

\begin{minipage}{12.5cm}
\begin{screen}
\begin{verbatim}
====================================
NOrbitalIdx 64   
ComplexType 0
====================================
====================================
   0     0     0 
   0     1     1 
   0     2     2 
   0     3     3 
 (continue...)
  15     9    62 
  15    10    63 
   0    1 
   1    1 
(continue...)
  62    1 
  63    1 
\end{verbatim}
\end{screen}
\end{minipage}

\subsubsection{ファイル形式}
以下のように行数に応じ異なる形式をとります($N_s$はサイト数、$N_{\rm o}$は変分パラメータの種類の数)。
 \begin{itemize}
   \item  1行:  ヘッダ(何が書かれても問題ありません)。
   \item  2行:   [string01]~[int01]
   \item  3行:   [string02]~[int02]
   \item  4-5行:  ヘッダ(何が書かれても問題ありません)。
   \item  6 - 5+$N_s^2$行: [int03]~[int04]~[int05]
   \item  6+$N_s^2$ - 5+$N_s^2+N_{\rm o}$行:[int06]~[int07]
  \end{itemize}
\subsubsection{パラメータ}
 \begin{itemize}

   \item  $[$string01$]$
   
    {\bf 形式 :} string型 (空白不可)

   {\bf 説明 :} 変分パラメータのセット総数のキーワード名を指定します(任意)。

   \item  $[$int01$]$
   
    {\bf 形式 :} int型 (空白不可)

   {\bf 説明 :} 変分パラメータのセット総数を指定します。

   \item  $[$string02$]$
   
    {\bf 形式 :} string型 (空白不可)

   {\bf 説明 :} 変分パラメータの型を指定するためのキーワード名を指定します(任意)。

   \item  $[$int02$]$
   
    {\bf 形式 :} int型 (空白不可)

   {\bf 説明 :} 変分パラメータの型を指定します。0が実数、1が複素数に対応します。

  \item  $[$int03$]$, $[$int04$]$
   
 {\bf 形式 :} int型 (空白不可)

{\bf 説明 :} サイト番号を指定する整数。0以上\verb|Nsite|{未満}で指定します。
 
 \item  $[$int05$]$
   
   {\bf 形式 :} int型 (空白不可)

  {\bf 説明 :} 変分パラメータの種類を表します。0以上[int01]{未満}で指定します。

 \item  $[$int06$]$
   
   {\bf 形式 :} int型 (空白不可)

  {\bf 説明 :} 変分パラメータの種類を表します(最適化有無の設定用)。0以上[int01]{未満}で指定します。
  
 \item  $[$int07$]$
   
   {\bf 形式 :} int型 (空白不可)

  {\bf 説明 :} [int06]で指定した変分パラメータの最適化有無を設定します。最適化する場合は1、最適化しない場合は0とします。
  
  
\end{itemize}

\subsubsection{使用ルール}
本ファイルを使用するにあたってのルールは以下の通りです。
\begin{itemize}
\item 行数固定で読み込みを行う為、ヘッダの省略はできません。
\item 成分が重複して指定された場合にはエラー終了します。
\item $[$int01$]$と定義されている変分パラメータの種類の総数が異なる場合はエラー終了します。
\item \tr{$[$int02$]$-$[$int07$]$を指定する際、範囲外の整数を指定した場合はエラー終了します。}
\end{itemize}



\newpage
\subsection{TransSym指定ファイル}
\label{Subsec:TransSym}

運動量射影${\cal L}_K=\frac{1}{N_s}\sum_{{\bm R}}e^{i {\bm K} \cdot{\bm R} } \hat{T}_{\bm R}$と格子対称性射影${\cal L}_P=\sum_{\alpha}p_{\alpha} \hat{G}_{\alpha}$について、重みとサイト番号に関する指定を行います。射影するパターンは$(\alpha, {\bm R})$で指定されます。\tr{射影を行わない場合も重み1.0 で“恒等演算”を指定してください。}
以下にファイル例を記載します。

\begin{minipage}{12.5cm}
\begin{screen}
\begin{verbatim}
====================================
NQPTrans 4  
====================================
== TrIdx_TrWeight_and_TrIdx_i_xi  ==
====================================
   0  1.000000
   1  1.000000
   2  1.000000
   3  1.000000
   0     0    0
 (continue...)
   3    12    1
   3    13    2 
\end{verbatim}
\end{screen}
\end{minipage}

\subsubsection{ファイル形式}
以下のように行数に応じ異なる形式をとります($N_s$はサイト数、$N_{\rm TS}$は射影演算子の種類の総数)。
 \begin{itemize}
   \item  1行:  ヘッダ(何が書かれても問題ありません)。
   \item  2行:   [string01]~[int01]
   \item  3-5行:  ヘッダ(何が書かれても問題ありません)。
   \item  6 - 5+$N_{\rm TS}$行: [int02]~[double01]
   \item  6+$N_{\rm TS}$ - 5+$(N_s+1) \times N_{\rm TS}$行:[int03]~[int04]~[int05]
  \end{itemize}
\subsubsection{パラメータ}
 \begin{itemize}

   \item  $[$string01$]$
   
    {\bf 形式 :} string型 (空白不可)

   {\bf 説明 :} 射影パターンの総数に関するキーワード名を指定します(任意)。

   \item  $[$int01$]$
   
    {\bf 形式 :} int型 (空白不可)

   {\bf 説明 :} 射影パターンの総数を指定します。

  \item  $[$int02$]$
   
 {\bf 形式 :} int型 (空白不可)

{\bf 説明 :} 射影パターン$(\alpha, {\bm R})$を指定する整数。0以上 $[$int01$]${未満}で指定します。
 
  \item  $[$double01$]$
   
 {\bf 形式 :} double型 (空白不可)

{\bf 説明 :} 射影パターン$(\alpha, {\bm R})$の重み$p_{\alpha}\cos ({\bm K}\cdot \alpha)$を指定します。
 
 \item  $[$int03$]$
   
   {\bf 形式 :} int型 (空白不可)

  {\bf 説明 :} 射影パターン$(\alpha, {\bm R})$を指定する整数。0以上 $[$int01$]${未満}で指定します。

 \item  $[$int04$]$, $[$int05$]$
   
   {\bf 形式 :} int型 (空白不可)

  {\bf 説明 :} サイト番号を指定する整数。0以上\verb|Nsite|{未満}で指定します。$[$int03$]$で指定した並進・点群移動をサイト番号$[$int04$]$に作用させた場合の行き先が、サイト番号$[$int05$]$となるように設定します。

\end{itemize}

\subsubsection{使用ルール}
本ファイルを使用するにあたってのルールは以下の通りです。
\begin{itemize}
\item 行数固定で読み込みを行う為、ヘッダの省略はできません。
\item 成分が重複して指定された場合にはエラー終了します。
\item $[$int01$]$と定義されている射影パターンの総数が異なる場合はエラー終了します。
\item \tr{$[$int02$]$-$[$int05$]$を指定する際、範囲外の整数を指定した場合はエラー終了します。}
\end{itemize}

\newpage
\subsection{InGutzwiller指定ファイル}
\label{Subsec:InGutzwiller}
\begin{equation}
{\cal P}_G=\exp\left[ \sum_i g_i n_{i\uparrow} n_{i\downarrow} \right]
\end{equation}
の$g_i$について初期値を設定します。

以下にファイル例を記載します。

\begin{minipage}{12.5cm}
\begin{screen}
\begin{verbatim}
======================
NGutzwillerIdx  16  
====================== 
===== Gutzwiller ===== 
====================== 
   0     0.0     0.0
   1     0.1     0.0
   2     0.0     0.0
   3     0.1     0.0
 (continue...)
  15     0.1     0.0
\end{verbatim}
\end{screen}
\end{minipage}

\subsubsection{ファイル形式}
以下のように行数に応じ異なる形式をとります($N_s$はサイト数)。
 \begin{itemize}
   \item  1行:  ヘッダ(何が書かれても問題ありません)。
   \item  2行:   [string01]~[int01]
   \item  3-5行:  ヘッダ(何が書かれても問題ありません)。
   \item  6 - 5+$N_s$行: [int02]~[double01]~[double02]
  \end{itemize}
\subsubsection{パラメータ}
 \begin{itemize}

   \item  $[$string01$]$
   
    {\bf 形式 :} string型 (空白不可)

   {\bf 説明 :} $g_i$の変分パラメータ総数のキーワード名を指定します(任意)。

   \item  $[$int01$]$
   
    {\bf 形式 :} int型 (空白不可)

   {\bf 説明 :} $g_i$の変分パラメータ総数を指定します。

  \item  $[$int02$]$
  
 {\bf 形式 :} int型 (空白不可)

{\bf 説明 :} サイト番号を指定する整数。0以上\verb|Nsite|{未満}で指定します。
 
 \item  $[$double01$]$
   
   {\bf 形式 :} double型 (空白不可)

  {\bf 説明 :} $g_i$の初期値の実部を与えます。
   
   \item  $[$double02$]$
   {\bf 形式 :} double型 

  {\bf 説明 :} $g_i$の初期値の虚部を与えます。\verb|Gutzwiller|指定ファイルで型を実数に指定している場合は、$[$double02$]$は入力しないでください。複素数指定の場合に$[$double02$]$がない場合には、0が代入されます。
   
\end{itemize}

\subsubsection{使用ルール}
本ファイルを使用するにあたってのルールは以下の通りです。
\begin{itemize}
\item 行数固定で読み込みを行う為、ヘッダの省略はできません。
\item 成分が重複して指定された場合にはエラー終了します。
\item $[$int01$]$と定義されている変分パラメータの総数が異なる場合はエラー終了します。
\item \verb|Gutzwiller|指定ファイルで型を実部に指定した状態で、$[$double02$]$が入力されるとエラー終了します。
\item \verb|Gutzwiller|指定ファイルで紐付けされるサイト番号とパラメータの種類と、入力ファイルで指定されるパラメータの値の整合性がとれない場合は警告を出します。その際、入力値としては平均された値が採用されます。例えば、\verb|Gutzwiller|指定ファイルで$i$, $j$サイトのパラメータの種類が共通の$0$に指定されているにも関わらず、入力ファイルで$i$,$j$サイトの値が異なる場合には警告が出されます。
\end{itemize}


\newpage
\subsection{InJastrow指定ファイル}
\label{Subsec:InJastrow}
Jastrow因子
\begin{equation}
{\cal P}_J=\exp\left[\frac{1}{2} \sum_{i\neq j} v_{ij} n_i n_j\right]
\end{equation}
の$v_{ij}$について初期値の設定を行います。

\begin{minipage}{12.5cm}
\begin{screen}
\begin{verbatim}
======================
NJastrowIdx 240 
====================== 
== i_j_JastrowIdx  ===
====================== 
   0     1     0.0     0.0 
   0     2     1.0     0.0
   0     3     0.0     0.0
 (continue...)
  16     14    0.0   0.0
  16     15    0.0   0.0
\end{verbatim}
\end{screen}
\end{minipage}

\subsubsection{ファイル形式}
以下のように行数に応じ異なる形式をとります($N_s$はサイト数)。
 \begin{itemize}
   \item  1行:  ヘッダ(何が書かれても問題ありません)。
   \item  2行:   [string01]~[int01]
   \item  3-5行:  ヘッダ(何が書かれても問題ありません)。
   \item  6 - 5+$N_s*(N_s-1)$行: [int02]~[int03]~[double01]~[double02]
  \end{itemize}
\subsubsection{パラメータ}
 \begin{itemize}

   \item  $[$string01$]$
   
    {\bf 形式 :} string型 (空白不可)

   {\bf 説明 :} $v_{ij}$の変分パラメータ総数のキーワード名を指定します(任意)。

   \item  $[$int01$]$
   
    {\bf 形式 :} int型 (空白不可)

   {\bf 説明 :} $v_{ij}$の変分パラメータ総数を指定します。

  \item  $[$int02$]$, $[$int03$]$
  
 {\bf 形式 :} int型 (空白不可)

{\bf 説明 :} サイト番号を指定する整数。0以上\verb|Nsite|{未満}で指定します。
 
 \item  $[$double01$]$
   
   {\bf 形式 :} double型 (空白不可)

  {\bf 説明 :} $v_{ij}$の初期値の実部を与えます。
  
 \item  $[$double02$]$
   
   {\bf 形式 :} double型 

  {\bf 説明 :} $v_{ij}$の初期値の虚部を与えます。\verb|Jastrow|指定ファイルで型を実数に指定している場合は、$[$double02$]$は入力しないでください。複素数指定の場合に$[$double02$]$がない場合には、0が代入されます。
  
\end{itemize}

\subsubsection{使用ルール}
本ファイルを使用するにあたってのルールは以下の通りです。
\begin{itemize}
\item 行数固定で読み込みを行う為、ヘッダの省略はできません。
\item 成分が重複して指定された場合にはエラー終了します。
\item $[$int01$]$と定義されている変分パラメータの総数が異なる場合はエラー終了します。
\item  \verb|Jastrow|指定ファイルで型を実数に指定した状態で、$[$double02$]$が入力されるとエラー終了します。
\item \verb|Jastrow|指定ファイルで紐付けされるサイト番号とパラメータの種類と、入力ファイルで指定されるパラメータの値の整合性がとれない場合は警告を出します。その際、入力値としては平均された値が採用されます。
\end{itemize}

\newpage
\subsection{InDH2指定ファイル}
\label{Subsec:InDH2}
\begin{equation}
{\cal P}_{d-h}^{(2)}= \exp \left[ \sum_t \sum_{n=0}^2 (\alpha_{2nt}^d \sum_{i}\xi_{i2nt}^d+\alpha_{2nt}^h \sum_{i}\xi_{i2nt}^h)\right]
\end{equation}
で表される2サイトのdoublon-holon相関因子の初期値設定を行います。指定するパラメータは$(n, t, s)$(s=0は中心がdoublon、1は中心がholon)に対応する番号と、$\alpha_{2nt}^{d,h}$の初期値です。以下にファイル例を記載します。

\begin{minipage}{12.5cm}
\begin{screen}
\begin{verbatim}
====================================
NDoublonHolon2siteIdx 32 
====================================
==  i_xi_xi_DoublonHolon2siteIdx  ==
====================================
   0    0.0    0.0
   1    0.0    0.0
   2    1.0    0.0
 (continue...)
  10   0.0    0.0
  11   1.0    0.0
\end{verbatim}
\end{screen}
\end{minipage}

\subsubsection{ファイル形式}
以下のように行数に応じ異なる形式をとります($N_s$はサイト数、$N_{\rm dh2}$は変分パラメータの種類の数)。
 \begin{itemize}
   \item  1行:  ヘッダ(何が書かれても問題ありません)。
   \item  2行:   [string01]~[int01]
   \item  3-5行:  ヘッダ(何が書かれても問題ありません)。
   \item  6 - 5+$N_{\rm dh2}$行: [int02]~[double01]~[double02]
  \end{itemize}
\subsubsection{パラメータ}
 \begin{itemize}

   \item  $[$string01$]$
   
    {\bf 形式 :} string型 (空白不可)

   {\bf 説明 :} 変分パラメータのセット総数のキーワード名を指定します(任意)。

   \item  $[$int01$]$
   
    {\bf 形式 :} int型 (空白不可)

   {\bf 説明 :} 変分パラメータのセット総数を指定します。

 \item  $[$int02$]$
   
   {\bf 形式 :} int型 (空白不可)

  {\bf 説明 :} \verb|DH2|指定ファイル内の変分プラメータの種類([int07])を指定します。値は0以上[int01]{未満}です。

 \item  $[$double01$]$
    
   {\bf 形式 :} double型 (空白不可)

  {\bf 説明 :} 変分パラメータの実部を与えます。
  
 
 \item $[$double02$]$
   
   {\bf 形式 :} double型

  {\bf 説明 :} 変分パラメータの虚部を与えます。 \verb|DH2|指定ファイルで型を実数に指定している場合は、$[$double02$]$は入力しないでください。複素数指定の場合に$[$double02$]$がない場合には、0が代入されます。
  
\end{itemize}

\subsubsection{使用ルール}
本ファイルを使用するにあたってのルールは以下の通りです。
\begin{itemize}
\item 行数固定で読み込みを行う為、ヘッダの省略はできません。
\item 成分が重複して指定された場合にはエラー終了します。
\item $[$int01$]$と定義されている変分パラメータの種類の総数が異なる場合はエラー終了します。
\item \verb|DH2|指定ファイルで型を実数に指定した状態で、$[$double02$]$が入力されるとエラー終了します。
\end{itemize}


\newpage
\subsection{InDH4指定ファイル}
\label{Subsec:InDH4}
\begin{equation}
{\cal P}_{d-h}^{(4)}= \exp \left[ \sum_t \sum_{n=0}^4 (\alpha_{4nt}^d \sum_{i}\xi_{i4nt}^d+\alpha_{4nt}^h \sum_{i}\xi_{i4nt}^h)\right]
\end{equation}
で表される4サイトのdoublon-holon相関因子の設定を行います。。指定するパラメータは$(n, t, s)$(s=0は中心がdoublon、1は中心がholon)に対応する番号と、$\alpha_{4nt}^{d,h}$の変分パラメータの初期値です。以下にファイル例を記載します。

\begin{minipage}{12.5cm}
\begin{screen}
\begin{verbatim}
====================================
NDoublonHolon4siteIdx 10
====================================
==  i_xi_xi_DoublonHolon4siteIdx  ==
====================================
   0     1.0     0.0
   1     0.0     0.0 
(continue...)
   8     1.0     0.0 
   9     0.0     0.0 
\end{verbatim}
\end{screen}
\end{minipage}

\subsubsection{ファイル形式}
以下のように行数に応じ異なる形式をとります($N_{\rm dh4}$は変分パラメータの総数)。
 \begin{itemize}
   \item  1行:  ヘッダ(何が書かれても問題ありません)。
   \item  2行:   [string01]~[int01]
   \item  3-5行:  ヘッダ(何が書かれても問題ありません)。
   \item  6 - 6+$N_{\rm dh4}$行:[int02]~[double01]~[double02]
  \end{itemize}
\subsubsection{パラメータ}
 \begin{itemize}

   \item  $[$string01$]$
   
    {\bf 形式 :} string型 (空白不可)

   {\bf 説明 :} 変分パラメータのセット総数のキーワード名を指定します(任意)。

   \item  $[$int01$]$
   
    {\bf 形式 :} int型 (空白不可)

   {\bf 説明 :} 変分パラメータのセット総数を指定します。

 \item  $[$int02$]$
   
   {\bf 形式 :} int型 (空白不可)

  {\bf 説明 :} \verb|DH4|指定ファイル内の変分プラメータの種類([int09])を指定します。値は0以上[int01]{未満}です。

 \item  $[$double01$]$
    
   {\bf 形式 :} double型 (空白不可)

  {\bf 説明 :} 変分パラメータの実部を与えます。
  
 
 \item $[$double02$]$
   
   {\bf 形式 :} double型

  {\bf 説明 :} 変分パラメータの虚部を与えます。 \verb|DH4|指定ファイルで型を実数に指定している場合は、$[$double02$]$は入力しないでください。複素数指定の場合に$[$double02$]$がない場合には、0が代入されます。
  
\end{itemize}

\subsubsection{使用ルール}
本ファイルを使用するにあたってのルールは以下の通りです。
\begin{itemize}
\item 行数固定で読み込みを行う為、ヘッダの省略はできません。
\item 成分が重複して指定された場合にはエラー終了します。
\item $[$int01$]$と定義されている変分パラメータの種類の総数が異なる場合はエラー終了します。
\item \verb|DH4|指定ファイルで型を実数に指定した状態で、$[$double02$]$が入力されるとエラー終了します。

\end{itemize}

\newpage
\subsection{InOrbital指定ファイル}
\label{Subsec:InOrbital}
\begin{equation}
|\phi_{\rm pair} \rangle = \left[\sum_{i, j=1}^{N_s} f_{ij}c_{i\uparrow}^{\dag}c_{j\downarrow}^{\dag} \right]^{N/2}|0 \rangle
\end{equation}
で表されるペア軌道の設定を行います。指定するパラメータはサイト番号$i, j$と変分パラメータ$f_{ij}$の初期値を設定します。以下にファイル例を記載します。

\begin{minipage}{12.5cm}
\begin{screen}
\begin{verbatim}
====================================
NOrbitalIdx 64  
====================================
==  i_j_OrbitalIdx  ==
====================================
   0     0     0.1    0.0 
   0     1     0.1    0.0     
   0     2     0.1    0.0    
   0     3     0.1    0.0    
 (continue...)
  15     9     0.2    0.0 
  15    10    0.2    0.0 
\end{verbatim}
\end{screen}
\end{minipage}

\subsubsection{ファイル形式}
以下のように行数に応じ異なる形式をとります($N_s$はサイト数)。
 \begin{itemize}
   \item  1行:  ヘッダ(何が書かれても問題ありません)。
   \item  2行:   [string01]~[int01]
   \item  3-5行:  ヘッダ(何が書かれても問題ありません)。
   \item  6 - 5+$N_s^2$行: [int02]~[int03]~[double01]~[double02]
  \end{itemize}
\subsubsection{パラメータ}
 \begin{itemize}

   \item  $[$string01$]$
   
    {\bf 形式 :} string型 (空白不可)

   {\bf 説明 :} 変分パラメータのセット総数のキーワード名を指定します(任意)。

   \item  $[$int01$]$
   
    {\bf 形式 :} int型 (空白不可)

   {\bf 説明 :} 変分パラメータのセット総数を指定します。

  \item  $[$int02$]$, $[$int03$]$
   
 {\bf 形式 :} int型 (空白不可)

{\bf 説明 :} サイト番号を指定する整数。0以上\verb|Nsite|{未満}で指定します。
 
  \item  $[$double01$]$
    
   {\bf 形式 :} double型 (空白不可)

  {\bf 説明 :} 変分パラメータの実部を与えます。
  
 
 \item $[$double02$]$
   
   {\bf 形式 :} double型

  {\bf 説明 :} 変分パラメータの虚部を与えます。 \verb|Orbital|指定ファイルで型を実数に指定している場合は、$[$double02$]$は入力しないでください。複素数指定の場合に$[$double02$]$がない場合には、0が代入されます。

  
\end{itemize}

\subsubsection{使用ルール}
本ファイルを使用するにあたってのルールは以下の通りです。
\begin{itemize}
\item 行数固定で読み込みを行う為、ヘッダの省略はできません。
\item 成分が重複して指定された場合にはエラー終了します。
\item $[$int01$]$と定義されている変分パラメータの種類の総数が異なる場合はエラー終了します。
\item $[$int02$]$, $[$int03$]$を指定する際、範囲外の整数を指定した場合はエラー終了します。
\item \verb|Orbital|指定ファイルで型を実数に指定した状態で、$[$double02$]$が入力されるとエラー終了します。
\item \verb|Orbital|指定ファイルで紐付けされるサイト番号とパラメータの種類と、入力ファイルで指定されるパラメータの値の整合性がとれない場合は警告を出します。その際、入力値としては平均された値が採用されます。
\end{itemize}



\newpage
\subsection{OneBodyG指定ファイル}
\label{Subsec:onebodyg}
一体グリーン関数$\langle c_{i\sigma_1}^{\dagger}c_{j\sigma_2}\rangle$を計算します。以下にファイル例を記載します。

\begin{minipage}{12.5cm}
\begin{screen}
\begin{verbatim}
===============================
NCisAjs         24
===============================
======== Green functions ======
===============================
    0     0     0     0
    0     1     0     1
    1     0     1     0
    1     1     1     1
    2     0     2     0
    2     1     2     1
    3     0     3     0
    3     1     3     1
    4     0     4     0
    4     1     4     1
    5     0     5     0
    5     1     5     1
    6     0     6     0
    6     1     6     1
    7     0     7     0
    7     1     7     1
    8     0     8     0
    8     1     8     1
    9     0     9     0
    9     1     9     1
   10     0    10     0
   10     1    10     1
   11     0    11     0
   11     1    11     1
\end{verbatim}
\end{screen}
\end{minipage}

\subsubsection{ファイル形式}
以下のように行数に応じ異なる形式をとります。
 \begin{itemize}
   \item  1行:  ヘッダ(何が書かれても問題ありません)。
   \item  2行:   [string01]~[int01]
   \item  3-5行:  ヘッダ(何が書かれても問題ありません)。
   \item  6行以降: [int02]~~[int03]~~[int04]~~[int05]
  \end{itemize}
\subsubsection{パラメータ}
 \begin{itemize}

   \item  $[$string01$]$
   
    {\bf 形式 :} string型 (空白不可)

   {\bf 説明 :} 一体グリーン関数成分総数のキーワード名を指定します(任意)。

   \item  $[$int01$]$
   
    {\bf 形式 :} int型 (空白不可)

   {\bf 説明 :} 一体グリーン関数成分の総数を指定します。

  \item  $[$int02$]$, $[$int04$]$

 {\bf 形式 :} int型 (空白不可)

{\bf 説明 :} サイト番号を指定する整数。0以上\verb|Nsite|{未満}で指定します。
 
  \item  $[$int03$]$, $[$int05$]$

 {\bf 形式 :} int型 (空白不可)

{\bf 説明 :} スピンを指定する整数。\\
0: アップスピン\\
1: ダウンスピン\\
を選択することが出来ます。

\end{itemize}

\subsubsection{使用ルール}
本ファイルを使用するにあたってのルールは以下の通りです。
\begin{itemize}
\item 行数固定で読み込みを行う為、ヘッダの省略はできません。
\item 成分が重複して指定された場合にはエラー終了します。
\item $[$int01$]$と定義されている一体グリーン関数成分の総数が異なる場合はエラー終了します。
\item $[$int02$]$-$[$int05$]$を指定する際、範囲外の整数を指定した場合はエラー終了します。
\end{itemize}

\newpage
\subsection{TwoBodyG指定ファイル}
\label{Subsec:twobodyg}
二体グリーン関数$\langle c_{i\sigma_1}^{\dagger}c_{j\sigma_2}c_{k\sigma_3}^{\dagger}c_{l\sigma_4}\rangle$を計算します。
{なお、スピンに関して計算する場合には、$i=j, k=l$となるよう設定してください。}
以下にファイル例を記載します。

\begin{minipage}{12.5cm}
\begin{screen}
\begin{verbatim}
=============================================
NCisAjsCktAltDC        576
=============================================
======== Green functions for Sq AND Nq ======
=============================================
    0     0     0     0     0     0     0     0
    0     0     0     0     0     1     0     1
    0     0     0     0     1     0     1     0
    0     0     0     0     1     1     1     1
    0     0     0     0     2     0     2     0
    0     0     0     0     2     1     2     1
    0     0     0     0     3     0     3     0
    0     0     0     0     3     1     3     1
    0     0     0     0     4     0     4     0
    0     0     0     0     4     1     4     1
    0     0     0     0     5     0     5     0
    0     0     0     0     5     1     5     1
    0     0     0     0     6     0     6     0
    0     0     0     0     6     1     6     1
    0     0     0     0     7     0     7     0
    0     0     0     0     7     1     7     1
    0     0     0     0     8     0     8     0
    0     0     0     0     8     1     8     1
    0     0     0     0     9     0     9     0
    0     0     0     0     9     1     9     1
    0     0     0     0    10     0    10     0
    0     0     0     0    10     1    10     1
    0     0     0     0    11     0    11     0
    0     0     0     0    11     1    11     1
    0     1     0     1     0     0     0     0
    …
\end{verbatim}
\end{screen}
\end{minipage}

\subsubsection{ファイル形式}
以下のように行数に応じ異なる形式をとります。
 \begin{itemize}
   \item  1行:  ヘッダ(何が書かれても問題ありません)。
   \item  2行:   [string01]~[int01]
   \item  3-5行:  ヘッダ(何が書かれても問題ありません)。
   \item  6行以降: [int02]~~[int03]~~[int04]~~[int05]~~[int06]~~[int07]~~[int08]~~[int09]
  \end{itemize}
\subsubsection{パラメータ}
 \begin{itemize}

   \item  $[$string01$]$
   
    {\bf 形式 :} string型 (空白不可)

   {\bf 説明 :} 二体グリーン関数成分総数のキーワード名を指定します(任意)。

   \item  $[$int01$]$
   
    {\bf 形式 :} int型 (空白不可)

   {\bf 説明 :} 二体グリーン関数成分の総数を指定します。

  \item  $[$int02$]$, $[$int04$]$,$[$int06$]$, $[$int08$]$

 {\bf 形式 :} int型 (空白不可)

{\bf 説明 :} サイト番号を指定する整数。0以上\verb|Nsite|{未満}で指定します。
 
  \item  $[$int03$]$, $[$int05$]$,$[$int07$]$, $[$int09$]$

 {\bf 形式 :} int型 (空白不可)

{\bf 説明 :} スピンを指定する整数。\\
0: アップスピン\\
1: ダウンスピン\\
を選択することが出来ます。

\end{itemize}

\subsubsection{使用ルール}
本ファイルを使用するにあたってのルールは以下の通りです。
\begin{itemize}
\item 行数固定で読み込みを行う為、ヘッダの省略はできません。
\item 成分が重複して指定された場合にはエラー終了します。
\item {スピンに関して計算する場合、$i=j, k=l$を満たさない場合ペアが存在するとエラー終了します。}
\item $[$int01$]$と定義されているニ体グリーン関数成分の総数が異なる場合はエラー終了します。
\item $[$int02$]$-$[$int09$]$を指定する際、範囲外の整数を指定した場合はエラー終了します。
\end{itemize}

\newpage
\section{出力ファイル}
\label{Sec:outputfile}
T.B.D
\newpage
\section{エラーメッセージ一覧}
T.B.D
%----------------------------------------------------------
%----------------------------------------------------------
%----------------------------------------------------------
