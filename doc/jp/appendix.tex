% !TEX root = userguide_jp.tex
%----------------------------------------------------------
\appendix
\chapter{非制限Hartree-Fock近似プログラム}
\label{Ch:URHF}
mVMCでは補助プログラムとして、多変数変分モンテカルロ法のペア軌道$f_{ij}$の初期値を非制限Hartree-Fock(UHF)近似から与えるためのプログラムを用意しています(対応関係は\ref{sec:PuffAndSlater}を参照)。なお、本プログラムは遍歴電子系を対象としており、スピン系、近藤系では正しく動作しません。

\section{概要}
UHF近似では揺らぎ$\delta A \equiv A-\langle A \rangle$の一次までを考慮することで、二体項を一体項へと近似します。
たとえば、サイト間クーロン相互作用
\begin{equation}
{\cal H}_V = \sum_{i,j, \sigma, \sigma'}V_{ij} n_ {i\sigma}n_{j\sigma'}
\end{equation}
について考えます。簡単化のため、$i\equiv (i, \sigma)$, $j\equiv (j, \sigma')$とすると相互作用の項は揺らぎの二次を落とすことで、
\begin{eqnarray}
n_ {i}n_{j} &=& (\langle n_{i} \rangle +\delta n_i) (\langle n_{j} \rangle +\delta n_j) - \left[ \langle c_{i}^{\dag}c_j \rangle +\delta (c_{i}^{\dag}c_j ) \right] \left[ \langle c_{j}^{\dag}c_i \rangle +\delta (c_{j}^{\dag}c_i )\right] \nonumber\\
&\sim&\langle n_{i} \rangle n_j+\langle n_{j} \rangle  n_i - \langle c_{i}^{\dag}c_j \rangle  c_{j}^{\dag}c_i  -  \langle c_{j}^{\dag}c_i \rangle c_{i}^{\dag}c_j 
-\langle n_{i} \rangle \langle n_j \rangle +  \langle c_{j}^{\dag}c_i \rangle \langle c_{i}^{\dag}c_j \rangle 
\end{eqnarray}
と近似されます。このような形式で、その他の相互作用についても近似を行うことで、一体問題に帰着させることができます。
計算では、上記の各平均値がself-consistentになるまで計算を行います。

\subsection{ソースコード}
ソースコード一式は\verb|src/ComplexUHF/src|以下に入っています。
\subsection{コンパイル方法}
コンパイルはmVMCのコンパイルと同様にmVMCのフォルダ直下で
\begin{verbatim}
$ make mvmc
\end{verbatim}
を実行することで行われます。コンパイルが終了すると、
\verb|src/ComplexUHF/src|に実行ファイル\verb|UHF|が作成されます。

\subsection{必要な入力ファイル}
\subsubsection{入力ファイル指定用ファイル (namelsit.def)}
UHFで指定するファイルは以下のファイルです。
\verb|namelist.def|は\ref{Subsec:InputFileList}で定義されているファイルと同じ様式です。\\
\begin{itemize}
\item{\verb|ModPara|}
\item{\verb|LocSpin|}
\item{\verb|Trans|}
\item{\verb|CoulombIntra|}
\item{\verb|CoulombInter|}
\item{\verb|Hund|}
\item{\verb|PairHop|}
\item{\verb|Exchange|}
\item{\verb|Orbital|}
\item{\verb|Initial|}
\end{itemize}
基本的にはmVMCと同じファイルとなりますが、
 \begin{itemize}
 \item{\verb|ModPara|ファイルで指定されるパラメータ}
 \item{\verb|Initial|ファイルの追加}
 \end{itemize}
がmVMCと異なります。以下、その詳細を記載します。


\subsubsection{ModParaファイルで指定するパラメータ}
UHFで指定するパラメータは以下のパラメータです。
\begin{itemize}
\item{\verb|Nsite|}
\item{\verb|Ne|}
\item{\verb|Mix|}
\item{\verb|EPS|}
\item{\verb|IterationMax|}
\end{itemize}
\verb|Nsite|, \verb|Ne|はmVMCと共通のパラメータで、以下の三つがUHF独特のパラメータです。
\begin{itemize}
\item{\verb|Mix|}\\
linear mixingをdouble型で指定します。mix=1とすると完全に新しいGreen関数に置き換えられます。
\item{\verb|EPS|}\\
収束判定条件をint型で指定します。新しく計算されたGreen関数と一つ前のGreen関数の残差が$10^{-\verb|eps|}$の場合に、計算が打ち切られます。
\item{\verb|IterationMax|}\\
ループの最大数をint型で指定します。
\end{itemize}
なお、mVMCで使用するその他パラメータが存在する場合はWarningが標準出力されます(計算は中断せずに実行されます)。

\subsubsection{Initialファイル}
グリーン関数$G_{ij\sigma_1\sigma_2}\equiv \langle c_{i\sigma_1}^\dag c_{j\sigma_2}\rangle$の初期値を与えます。
ファイル様式は\verb|Trans|ファイルと同じで、$t_{ij\sigma_1\sigma_2}$の代わりに$G_{ij\sigma_1\sigma_2}$の値を記述します。
なお、値を指定しないグリーン関数には0が入ります。

\section{使用方法}
UHF自体はmVMCと同じように
\begin{verbatim}
$ UHF namelist.def
\end{verbatim}
で動きます。計算の流れは以下の通りです。
\begin{enumerate}
\item{ファイル読み込み}
\item{ハミルトニアンの作成}
\item{グリーン関数の計算 (self-consistentになるまで)}
\item{$f_{ij}$、各種ファイルの出力}
\end{enumerate}
計算後に出力されるファイルは以下の通りです。
\begin{itemize}
\item{zvo\_result.dat:}  energyと粒子数
\item{zvo\_check.dat:} 収束過程のエネルギー、残差
\item{zvo\_UHF\_cisajs.dat:} 収束したGreen関数
\item{zvo\_eigen.dat:} 収束したハミルトニアンの固有値
\item{zvo\_gap.dat:} エネルギーギャップ
\item{zvo\_orbital\_opt.dat:} スレータ行列式から生成した$f_{ij}$
\end{itemize}
なお、 zvo\_orbital\_opt.dat:は\verb|Orbital|ファイルを参照し、該当する$f_{ij}$について計算されます。
なお、値については\verb|Orbital|ファイルで同種と定義されたパラメータについて、
UHFで平均化した値が採用されます。



%----------------------------------------------------------
